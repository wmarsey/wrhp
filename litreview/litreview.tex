\documentclass[a4paper,11pt,twoside,notitlepage]{article}
\usepackage[left=2.5cm,right=2cm,top=2cm,bottom=2cm]{geometry}
\usepackage{graphicx}
\usepackage[colorlinks=false, pdfborder={0 0 0}]{hyperref}
\usepackage{cleveref} 
\usepackage{fancyhdr}
\usepackage{abstract}
\usepackage{framed}
\usepackage{cite}
\usepackage{amsmath}
\usepackage{natbib}
\usepackage{parskip}
\usepackage[procnames]{listings}
\usepackage{color}
\usepackage{appendix}
\usepackage{enumitem}
\setitemize{noitemsep,topsep=0pt,parsep=0pt,partopsep=0pt}

\definecolor{keywords}{RGB}{255,0,90}
\definecolor{comments}{RGB}{0,0,113}
\definecolor{red}{RGB}{160,0,0}
\definecolor{green}{RGB}{0,150,0}

\lstset{frame=tb,
  language=Python,
  aboveskip=3mm,
  belowskip=3mm,
  showstringspaces=false,
  columns=flexible,
  basicstyle={\small\ttfamily},
  numbers=none,
  numberstyle=\tiny\color{gray},
  keywordstyle=\color{keywords},
  commentstyle=\color{comments},
  stringstyle=\color{red},
  breaklines=true,
  breakatwhitespace=true,
  tabsize=3,
  procnamekeys={def,class}
}

%% BOXED ABSTRACT
\renewenvironment{abstract}
 {
	\small
  	\begin{center}
  	\bfseries \abstractname\vspace{-.5em}\vspace{0pt}
  	\end{center}
  	\list{}{
    	\setlength{\leftmargin}{.5cm}%
    	\setlength{\rightmargin}{\leftmargin}%
  	}%
  	\item\relax}
 	{\endlist}

%% OK LETS GO I GUESS
\begin{document}
	\title{Contribution
		\\ \small Literature Review}
	\author{William Marsey
		\\Imperial College London}
	\date{June 2014}
 	\maketitle
	

	%% \begin{framed}
	%% 	\begin{abstract}
	
	%% 	Mediawiki's slogan reads: ``Mediawiki: Because ideas
        %%         want to be free''. 

        %%         But qualitatively assessing these articles may be
        %%         something of a moot point. 
	
	%% 	%	Traditionally, a collaborative work may not
        %%            %     disclosed the provenance of its constituent
        %%            %     parts. An individual's stake in the
        %%            %     final work - the valued work - was not
        %%            %     apparent or accounted for in the final
        %%             %    text. Regardless of how clearly demarcated the various
        %%             %    roles in creating the artefact, an individual
        %%             %    collaborator's stake may have been
        %%             %    gauranteed only by way of prior
        %%             %    negotiation. These stakeholder policies were individual
        %%             %    and capricious, and bound the completeness of the work,
        %%             %    the finality (or authenticity) of the
        %%             %    work. 
	%% 	\end{abstract}
	%% \end{framed}

        \section{Introduction}
        \subsection{The Research Question}
        Given a collaboratively edited document, and information about
        who edited the document, and in what way, how do we measure
        the quality of their contribution?

        Colloborative work is becoming a big deal. It is both
        interesting and an important trend in modern computer
        use. And the data is abundant. 

        Amongst many other things, this topic is a playground for
        sociology, machine learning, network studies, as well as more
        general studies of conflict, and personality. My work intends
        to focus on the algorithmic side of things - approximate
        string matching in particular. I look at how we may use
        Levenshtein distance, and the various favours,
        varieties and optimizations thereof, to measure contribution
        to a collaborative text, and how we may implement a version of
        this algorithm specifically tailored to our needs.

        The main questions we ask are:
        \begin{itemize}
          \item What does Levenshtein distance define of a
            contribution in the context of massive online
            collaboration?
          \item What are the limitations and implications of defining
            contribution in this way?
          \item What else may we learn from analysing contribution? 
        \end{itemize}

        We base our studies around data from Wikipedia. This study is
        defined by -- and in some ways determined by -- the specific
        context of Wikipedia, but, as we will see, is ultimately
        enriched by it. Due to its open-source nature, and its size,
        studies that touch upon Wikipedia cover a very broad range of
        topics. Many of them are directly related to the topic we
        concern ourselves with here, and many more may enrich our
        study tangentially. 
 
        \section{Previous work}
        discuss your sources
        by theme, then date

        \subsection{Wikipedia}
        identify, analyse

        emblem of the Web2.0 era \cite{Mesgari2014}

        used to predict box office success \cite{Mestyan2012} But does
        it reveal westernness?

        Denning: Wikipedia risks: Accuracy, Motives, Uncertain Expertise,
        Volatility, Coverage, Sources (not many offline sources)
        \cite{Denning2005} 

        Denning says it cannot attain the status of a true
        encyclopedia without more formal content-inclusion and expert
        review procedures\cite{Denning2005} this corroborates by
        findings in \cite{Giles2005}?

        \subsubsection{On Wikipedia}
        `robust and remarkable growth'
        \cite{Kittur2007}\cite{Voss2005} 
        
        Wikipedia, at the last dump, consisted 800G of compressed data
        \cite{wiki-dump}

        \subsubsection{Evaluating Wikipedia articles}
        identify, analyse

        after article mentioned in press \cite{Lih2004}

        compared by 'experts' to 'equivalent' Encyclopedia Britannica articles \cite{Giles2005}

        found metrics of article quality through factor analysis
        \cite{Stvilia2005}

        Analysis by conflict - revisions?\cite{Kittur2007}

        WikiTrust. The most `complete' of the many of the. Exists as
        firefox plugin (though it doesn't work any more) Culmination
        of various studies that try to QUOTE \cite{Adler2007} and QUOTE CITE. It
        was assessed as recently as 2011 \cite{Lucassen2011}
       
        \subsection{Edit difference}

        Standard: Levenshtein distance \cite{Levenshtein1966}
        
        \subsection{My work in context of these sources}
        different views
        emerging topics
        gaps and inconsistencies

        \section{Conclusions}

        PREDIFINED / NOT-PREDEFINED ideas of quality. look for when
        the the article levels off? And do this by DATE rather than
        REVISION. We may assume that pageviews are more
        well-distributed than revisions

        summarize major contributions (which do we care about?)
        evaluate your current position
        point out any flaw in methodology/research/contradictions
        are there any gaps in the area which you will cover in your research?
        How will you integrate sources you have mentioned into your dissertation?
        Point out any areas for further study

\clearpage
\begin{appendices}
\section{Appendix A: Python class for scraping Wikipedia article version}
\subsection{Code}
\lstinputlisting[language=Python,frame=single]{../wikiScraper/WikiRevisionScrape.py}
\subsection{Example output}

\clearpage
\section{Appendix B: Python class for basic, space-naive Levenshtein
  implementation}
\subsection{Code}
\lstinputlisting[language=Python,frame=single]{../lshtein/basic/LevDistBasic.py}
\subsection{Example output}
\end{appendices}

\bibliography{litreview}{}
\bibliographystyle{plain}	
\end{document}
