\documentclass[a4paper,11pt,twoside,notitlepage]{article}
\usepackage[left=2.5cm,right=2cm,top=2cm,bottom=2cm]{geometry}
\usepackage{graphicx}
\usepackage[colorlinks=false, pdfborder={0 0 0}]{hyperref}
\usepackage{cleveref} 
\usepackage{fancyhdr}
\usepackage{abstract}
\usepackage{framed}
\usepackage{cite}
\usepackage{amsmath}
\usepackage{natbib}

\renewenvironment{abstract}
 {
	\small
  	\begin{center}
  	\bfseries \abstractname\vspace{-.5em}\vspace{0pt}
  	\end{center}
  	\list{}{
    	\setlength{\leftmargin}{.5cm}%
    	\setlength{\rightmargin}{\leftmargin}%
  	}%
  	\item\relax}
 	{\endlist}


\begin{document}
	\title{Chao Wu
		\\ \small Literature Review}
	\author{William Marsey
		\\Imperial College London}
	\date{June 2014}
 	\maketitle
	

	%% \begin{framed}
	%% 	\begin{abstract}
	
	%% 	Mediawiki's slogan reads: ``Mediawiki: Because ideas
        %%         want to be free''. 

        %%         But qualitatively assessing these articles may be
        %%         something of a moot point. 
	
	%% 	%	Traditionally, a collaborative work may not
        %%            %     disclosed the provenance of its constituent
        %%            %     parts. An individual's stake in the
        %%            %     final work - the valued work - was not
        %%            %     apparent or accounted for in the final
        %%             %    text. Regardless of how clearly demarcated the various
        %%             %    roles in creating the artefact, an individual
        %%             %    collaborator's stake may have been
        %%             %    gauranteed only by way of prior
        %%             %    negotiation. These stakeholder policies were individual
        %%             %    and capricious, and bound the completeness of the work,
        %%             %    the finality (or authenticity) of the
        %%             %    work. 
	%% 	\end{abstract}
	%% \end{framed}

        \section{Introduction}
        \subsection{The Research Question}
        (what is the main question/s you are trying to answer in your dissertation?)
        Define topic/Why this topic?
        Point out overall trends, gaps, themes

        %% \section{Wikipedia}
        %% It is worth giving a very brief overview of Wikipedia. In
        %% particular, it is worth decribing the opportunities and
        %% limitations of the implementation of Mediawiki software which
        %% is its foundation. 

        \section{Previous work}
        discuss your sources
        by theme, then date

        \subsection{Wikipedia}
        identify, analyse

        Denning: Wikipedia risks: Accuracy, Motives, Uncertain Expertise,
        Volatility, Coverage, Sources (not many offline sources)
        \cite{Denning2005} 

        Denning says it cannot attain the status of a true
        encyclopedia without more formal content-inclusion and expert
        review procedures\cite{Denning2005} this corroborates by
        findings in \cite{Giles2005}?

        \subsubsection{On Wikipedia}
        `robust and remarkable growth'
        \cite{Kittur2007}\cite{Voss2005} 
        
        Wikipedia, at the last dump, consisted 800G of compressed data
        \cite{wiki-dump}

        \subsubsection{Evaluating Wikipedia articles}
        identify, analyse

        after article mentioned in press \cite{Lih2004}

        compared by 'experts' to 'equivalent' Encyclopedia Britannica articles \cite{Giles2005}

        found metrics of article quality through factor analysis
        \cite{Stvilia2005}

        Analysis by conflict - revisions?\cite{Kittur2007}

        WikiTrust. The most `complete' of the many of the. Exists as
        firefox plugin (though it doesn't work any more) Culmination
        of various studies that try to QUOTE \cite{Adler2007} and QUOTE CITE. It
        was assessed as recently as 2011 \cite{Lucassen2011}
       
        \subsection{Edit difference}

        Standard: Levenshtein distance \cite{Levenshtein1966}
        
        \subsection{My work in context of these sources}
        different views
        emerging topics
        gaps and inconsistencies

        \section{Conclusions}

        PREDIFINED / NOT-PREDEFINED ideas of quality. look for when
        the the article levels off? And do this by DATE rather than
        REVISION. We may assume that pageviews are more
        well-distributed than revisions

        summarize major contributions (which do we care about?)
        evaluate your current position
        point out any flaw in methodology/research/contradictions
        are there any gaps in the area which you will cover in your research?
        How will you integrate sources you have mentioned into your dissertation?
        Point out any areas for further study

        %% \section{Wikipedia}
        %% \subsection{Brief overview}

        %% \subsection{Native valuation systems}
        %% Wikipedia articles are rated on an article-by-article basis by
        %% groups of volunteers, to a specification detailed on the
        %% Wikipedia itself. There are numerous grades with various
        %% qualifying characteristics, and many groups of volunteers - or
        %% 'Wikiprojects' - may rate the same article. A bot
        %% automatically gathers and averages these ratings, so that they
        %% may guide editors in improving all articles to a state of
        %% 'completion'. (It is important to remember that due to the
        %% nature of Wikipedia that this is 'completion' in a qualitative
        %% sense, and certainly does not allude to finality. There are
        %% many states of completion.)

        %% As with most other aspects of Wikipedia, this scale system is
        %% negotiated between various different parties and is, in
        %% itself, non-standardised. The Wikipedia article on the
        %% standard has itself a section named 'Non-standard' grades.

        %% \subsection{Technological Limitations}
        %% The wikipedia article that ranks  Wikipedia users by their
        %% number of edits has a caveat lector ('user beware')
        %% section. It warns readers of the misguided conclusions that
        %% can be easily taken by reading these numbers, and serves as a
        %% good guide for this project as to what native data to trust.

        %% In this project, if we use this data at all, we will mine
        %% afresh.

        %% \subsection{Previous work}
        %% \subsubsection{McGuinness, et al.}
        %% Evaluating articles according to internal and external
        %% links. This may be an appropriate strategy, as one study has
        %% shown this to be a natural heuristic for the average Wikipedia
        %% reader.~cite{Lucassen2010}

        %% \subsubsection{WikiTrust, and others} 

        %% A benchmark piece of software is WikiTrust. It was developed
        %% in response to Wikipedia reaching a milestone of maturity
        %% (having by 2006 collected content comparable to Encyclopedia
        %% Britannica [CITE GILES 2005]), and designed to evaluate each
        %% word of an article for its 'trustworthiness'. It also
        %% developed out research made by others, and incorporated a few of
        %% the research and heuristics gathered by others, such as the
        %% [SUMMARY LATER] of Kittur, et al., and the [SUMMARY LATER] of
        %% Cross. 

        %% [CRITICISMS in Luyt et al.]

        %% This trust is expressed as a real number between 0 and 1, and
        %% calculated using the 'age' of the word (each edit that word
        %% survives is regarded implicit attestment to worth on behalf of
        %% the editor), and the reputation of the contributor of that
        %% word. Conversely, the reputation of a contributor is
        %% calculated using the survival rate of his words. The 'trust'
        %% of each word is then represented as a grade of yellow,
        %% highlighting the word as we see below - the darker yellow
        %% parts are the most trusted parts.

        %% [DIAGRAM: WIKITRUST IN ACTION OMG]

        %% The software, although a little aged now (the last update was
        %% in 2008), tackles a lot of the concepts we concern ourselves
        %% with here, as well as guiding us to solutions to a few of the
        %% problems we will come accross, such as different kinds of
        %% malicious input. There has been a critical evaluation of it as
        %% recent as 2011, \cite{Lucassen2011}. 

        %% [TALK ABOUT HOW TRUST IS ONLY ONE PART OF VALUE]

        %% [TALK ABOUT how lucassen found the software to be kind of
        %%   useless, but a nice thing anyway, but basically useless, lol]
 
	%% \section{Tackling the Edit Distance Problem}
        %% Much of the greatest algorithm discoveries in this area come
        %% out of work made in the 60s and 70s. We find Levenshtein's
        %% work, and his eponymous edit distance algorithm to be the nexus of
        %% this work~/cite{Levenshtein1966} - much of the work in the 50 years since the
        %% publication of Levenshtein's original work has been working
        %% towards confirming and improving this original algorithm. \cite{Navarro2001}

        %% This family of algorithms tackle the problem in a similar way
        %% - comparing each of the characters of two strings, and
        %% creating some sort of table with which to speed up computation
        %% of edit distance. These improvements are manifold, but
        %% cluster around two principles: disregard rows and columns of
        %% the table that we know not to improve the final answer;
        %% compute the comparisons in a way that leverages the computer's
        %% natively quick operations (i.e. bit-wise comparisons,
        %% bitmapping representations of the strings, etc).

        %% The algorthims are principally Dynamic programming algorithms
        %% - the space complexities for most are usually the limiting
        %% factors in terms of implementation. Navarro notes that one of
        %% the quickest algorithms in this area is 'only of theoretical
        %% interest', having a space requirement [OF LOADS].

        %% \subsection{Intentions}
        %% \subsubsection{An examination of quality assessment in massive
        %%   collaborations}
        %% [REFER to Dalip, Wiki case study - 'new type of repository for
        %%   human knowledge'] ['12 million articles, written in around 
        %%   one hundred languages']

        %% One main axiom - good quality MUST be assumed to be the same
        %% as edit survival. We must assume, in this context, that
        %% collective agreement. It is part of the axiom on which
        %% wikipedia's operate. [from the dalip, read on all these things
        %% he references]


\bibliography{litreview}{}
\bibliographystyle{plain}	
\end{document}
