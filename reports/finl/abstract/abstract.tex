\addcontentsline{toc}{chapter}{Abstract}

\begin{abstract}
In this project, we analyse Wikipedia revision histories, with a view
to automatically deriving an collaborative share for each of the
article's editors. We use automatic, semantically-na\"ive string
analysis as a basis for discussing the value of each contribution in
an article's history. We analyse the edit distance between texts in
various ways, and, in particular, try to characterise the each step of
an article's path as connected to, completely contextualised by, every
other in that history.

As a Wikipedia article is something that evolves organically out of
the actions of many competing agents, then we may measure the success
of a contribution -- the smallest unit of which is a character -- to
be it's survival rate. Other studies have remarked upon this
also. However, previous studies that endeavour to characterise or
measure this survival rate concentrate on identifying binary do/undo
interaction between edits, which -- as we'll see -- is a common
occurrence, but we feel is inadequate in terms of really understanding
the constant, nuanced to-and-fro of mass on-line collaboration. We
offer a solution in our `trajectory' technique.

We also consider the text as composed of numerous different species of
text, dividing it using Wikipedia's native mark-up convention,
`wikimarkup', and look to find correlations between text species and
survival rate.

We find that, on Wikipedia in particular, that qualitative analysis of
text, at least characterised by species in this way, does not
correlate well to survival.

Though this observation may render a fair amount of the text-analysis
work obsolete in the case of analysing Wikipedia in particular, the
produced software is robust and malleable, and the procedures
developed and implemented here would lend themselves well to analysing
the interaction-driven analytical paradigm aforementioned. As it
stands, this work would adapt well to analysis of more stable forms of
on-line contribution, such as source control models.
\end{abstract}
