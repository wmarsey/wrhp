\addcontentsline{toc}{chapter}{Abstract}

\begin{abstract}

This project analyses Wikipedia revision histories, taking automatic
string analysis as a basis for discussing the value of each edit, and
each contribution in an article's history. We analyse the edit
distance between texts in various ways, and, in particular, try to
characterise the each step of an article's path as connected to,
completely contextualised by, every other in that history.

It strikes us that, as a Wikipedia article is something that evolves
organically out of the actions of many competing agents, then we may
measure the success of a contribution -- the smallest unit of which is
a character -- to be it's survival rate. Other studies have remarked
upon this also. However, previous studies that endeavour to
characterise or measure this survival rate concentrate on identifying
binary do/undo interaction between edits, which -- as we'll see -- is
a common occurence, but we feel is inadequate in terms of really
understanding the constant, nuanced to-and-fro of mass online
collaboration.

As well as analysing each edit in the context of every other, we spend
time exploring value in the fact of the edit itself -- may we evaluate
an edit according to it's actual content? We consider the text as
composed of numerous different species of text, dividing it using
Wikipedia's native markup convention, `wikimarkup', and look to find
correlations between text species and survival rate. Indeed this
project follows one that did only this, and most of the time spent on
this project was automating and exploring this approach.

We find that, on Wikipedia in particular, that qualitative analysis of
text, at least characterised by species in this way, does not
correlate well to survival, and thus to any kind of Wikipedia-mediated
quality. In the chaos of online interaction that characterises
Wikipedia, one of the most popular websites in the world, we find that
other factors are more correlated with survival -- factors incident to
to the nature of the Wikipedia `community', and a user's place within
that community, rather than the fact of the edit itself. We also
discuss complicating external factors.

Though this observation may render a fair amount of the text-analysis
work obsolete in the case of analysing Wikipedia in particular, the
produced software is robust and malleable, and the techniques used and
implemented here would lend themselves well to analysing the
network-driven and interaction-driven analystical paradigm formerly
aforementioned. And, as it stands, this work would adapt well to
analysis of more stable forms of online contribution, such as source
control models. We end by discussing and demonstrating the prospects
these possible extensions.
\end{abstract}
