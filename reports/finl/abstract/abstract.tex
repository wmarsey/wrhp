\addcontentsline{toc}{chapter}{Abstract}

\begin{abstract}

This project explores the measurement of a Wikipedia revision, and by
extension, mesaurement of Wikipedia. Using a textbook Levenshtein
distance algorithm as its basis, the algorithm outlined here allows us
to describe an edit in terms of various appropriate metrics. The end
product is a model which allows us to examine each edit in terms of
the species of the text that was edited, and in terms of its relation
to the rest of the edits in that Wikipedia articles history. We can
use the same model to examine the nature of editing in that article,
whether it was edited frequently, whether the article's responds well
to new editors, etc. 

While creating this model we discover a few interesting things about
the nature of Wikipedia, the editor community that control it, and
their regard to content quality, as well as various things to
understand about using Wikipedia as a data source for study. We find,
very interestingly, that an edit's acceptance into the article may not
necessarily be determined by the content of the edit, and that perhaps
consideration of an edit may not need to concern the text content at
all.

\end{abstract}
