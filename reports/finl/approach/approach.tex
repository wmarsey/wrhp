\chapter{Our approach}

\section{Data collection and storage}
\subsection*{The WikiMedia API}
WikiMedia's API service provides simple access to a wiki data,
features and meta-data over HTTP,\cite{wiki-api} and for this project,
provides the entirety of our data. In this section we explore the
process of collecting Wikipedia data, the peculiarities of Wikipedia
as a data source, and the algorithmic caveats necessary to deal with
them.

Our basic request is simple: we send a HTTP request to a given wiki
site's `/api.php' file, sending the query parameters a
`prop=revisions', `rvprop=content', and `titles=X|Y' to get those
page's most recent revision contents. We can prefetch a title using a
random request (parameters `list=random\&rnlimit=1'), and in each case
we can add the `format=json' parameter, and quite easily parse the
results. 

To trace the history of a given page, then, we need only augment the
`rvprop' argument to include `ids' (`rvprop=content|ids') in order to
discover the parent id and trace the history backwards from there. The
procedure defined in algorithm~\ref{alg-data} demonstrates this
clearly, showing the child-parent swap at line~\ref{datal2}. The
condition on the while loop (line~\ref{datal1}), however, shows the
first oddity with the Wikipedia histories. Most articles will
terminate at their origin, showing parentid 0. Some, instead, enter a
terminal cycle, with the oldest fetched revision giving its parent to
be a much younger version of the same article. 

%%%% Data fetching algorithm
\begin{algorithm}
  \caption{Data fetching}\label{alg-data}
  \begin{algorithmic}
    \Procedure{Fetch}{$pageid$}
    \State $corrupt \gets \emptyset$
    \State $visitedpages \gets \emptyset$
    \State $revid \gets 0$
    \State $parentid \gets wiki.getlatest(pageid)$
    \While{$revid \ne 0$ AND $revid \notin visitedpages$}\label{datal1} 
    \If{$revid$ is in the database}
    \State $parentid \gets database.getparentid(revid)$
    \Else
    \State $pagedata \gets wiki.getpage(revid)$
    \EndIf
    \If{$pagedata$ is corrupt}\label{datal3}
    \If{corruptness is within recoverability bounds}
    \State $corruptpages \gets corruptpages + (revid, parentid, domain)$
    \Else
    \State terminate fetch
    \EndIf
    \Else
    \State $database \gets page data$
    \EndIf
    \State $visitedpages \gets visitedpages + (revid, domain)$
    \State $revid \gets parentid$\label{datal2}
    \EndWhile
    \ForAll {$(revision, parent, domain) \in corrupt$}
    \State $CorruptClean(revision, parent, domain)$\Comment{See
      algorithm~\ref{corrupt-clean}}
    \EndFor
    \State Mark $pageid$ as complete in $database$
    \EndProcedure
  \end{algorithmic}
\end{algorithm}

We must do is keep a track of all the revisions we already know exist
of that article, so that we may identify these cycles, and terminate
the fetch loop early. In early tests, we find that these cycles occur
exclusively amongst older articles (though not all old articles have
the problem). For the purposes of this study, then, we must
acknowledge that any `complete' history of an article in our databases
is in fact only the complete \textit{discoverable}
history.\footnote{We may note that the Wikipedia site warns that some
  histories are only part-discoverable. Using our fetch algorithm we
  have managed to often surpass the boundaries described on Wikipedia,
  though the cycle problem does occur at a time approximate to that
  mentioned on the Wikipedia. We may assume that some server
  corruption occurred at around that point in Wikipedia's history.}

A second notable problem was the regular corrupt values returned by
Wikipedia. We see that we test for corruption in at line~\ref{datal3},
storing a list of corrupt pages - often the data is returned with
missing entries, but was often missing data that was pertinent to
later study - namely, timestamp and revision content. For ease of
analysis later, we choose to `circumnavigate' corrupt entries in the
database, changing pointers between children and parents in situ. We
may then later trace a history of incorrupt entries using these
pointers. The procedure for this is detailed in
algorithm~\ref{corrupt-clean}.

Finally, we define some corruptness of data to constitute a fetching
failure. With our model, we only do this when the parentid is
missing. Though this does not happen often, to handle these cases we
mark a pageid as having a succesful fetch by adding it to a special
table in the database. This can also be useful for fetches that are
interrupted in other ways, as with hardware and network problems. This
table is found in figure~\ref{database-schema} as `wikifetch'. 

\begin{algorithm}
  \caption{Corrupt pages}\label{corrupt-clean}
  \begin{algorithmic}
    \Procedure{CorruptClean}{corruptrev, parent, domain}
    \State $childrev \gets database.getchild(corruptrev)$
    \State $database.setparent(childrev, parent)$ \Comment{Now the
      corrupt revid is circumnavigated}
    \EndProcedure
  \end{algorithmic}
\end{algorithm}

\subsection*{The database}
The database is implemented in PSQL, and accessed via a python package
leveraging the psycopg library. The package is used by all the other
classes used in this project and provides simple inserting, changing,
fetching and checking of data. The only more complex operation is the
datadump function. The operation is more specific than the other
functions, and may have been pieced together using these functions,
but in implementation it was much quicker to correlate and fetch data
using the SQL `JOIN' statements, rather than multiple fetches in a
Python for loop, for example.

The database relies upon the uniqueness of revision ID and domain-name
pairs to function. The database schemata can be found in
figure~\ref{database-schema}. These databases are referenced at all
points of this project.

\begin{figure}
  \label{database-schema}
  \centering
  \begin{subfigure}[b!]{0.3\linewidth}
    \centering
    \begin{tabular}{ccc}
      \toprule
      \underline{revid} & \underline{domain} & content\\
      \midrule
      $\vdots$ & $\vdots$ & $\vdots$\\
    \end{tabular}
    \caption{Table: wikicontent}
  \end{subfigure}
  \begin{subfigure}[b!]{0.3\linewidth}
    \centering
    \begin{tabular}{cc}
      \toprule
      \underline{pageid} & \underline{domain} \\
      \midrule
      $\vdots$ & $\vdots$\\
    \end{tabular}
    \caption{Table: wikifetched}
  \end{subfigure}
  \begin{subfigure}[b!]{0.3\linewidth}
    \centering
    \begin{tabular}{cccc}
      \toprule
      \underline{revid1} & \underline{revid2} & \underline{domain} & distance\\
      \midrule
      $\vdots$ & $\vdots$ & $\vdots$ & $\vdots$ \\
    \end{tabular}
    \caption{Table: wikitrajectory}
  \end{subfigure}\\
  \vspace{10 mm}
  \begin{subfigure}[b!]{\linewidth}
    \centering
    \begin{tabular}{ccccccccc}
      \toprule
      \underline{revid} & \underline{domain} & pageid & title & username & userid & time & size &
      comment \\ 
      \midrule
      $\vdots$ & $\vdots$ & $\vdots$ & $\vdots$ & $\vdots$ & $\vdots$ & $\vdots$
      & $\vdots$ & $\vdots$ \\
    \end{tabular}
    \caption{Table: wikirevisions}
  \end{subfigure}

  \begin{subfigure}[b!]{\linewidth}
    \centering
    \begin{tabular}{ccccccccc}
      \toprule
      \underline{revid} & \underline{domain} & maths & citations & filesimages & links &
      structure & normal & gradient\\
      \midrule
      $\vdots$ & $\vdots$ & $\vdots$ & $\vdots$ & $\vdots$ & $\vdots$ &
      $\vdots$ & $\vdots$ & $\vdots$ \\
    \end{tabular}
    \caption{Table: wikiweights} 
    \subref{weightstable}
  \end{subfigure}
  \caption{Schemata for the database used to store wikipedia data}
\end{figure}

\section{Analytical procedures}

The data we collect via the Wikipedia API goes through a series of
procedures in order to extract measurements of it.

%pair comparison / weighted distance
First we compare the difference between the child-parent pairs of
revisions. This process is fairly simple, and is described in
algorithm~\ref{pair-comp}. The only special condition we introduce
here is comparing the oldest revision with an empty string.

%%%%% Pair comparison algorithm
\begin{algorithm}
\caption{Pair comparison}\label{pair-comp}
  \begin{algorithmic}
    \Procedure{PairComparison}{$revids$}
    \For {$ i \gets 0, $length($revids$)}
    \If{pair distance not already in database}
    \State $str1 \gets $\LineIf{``"}{$i=0$}{database.gettext($revs[i-1]$)}
    \State $str2 \gets $database.gettext($revs[i]$)
    \State $dist \gets $PairDistance($str1, str2$)\Comment{See algorithm~\ref{dist-calc}}
    \State databaseinsert.pairdistanceinsert($dist$)  
    \EndIf
    \EndFor
  \EndProcedure
  \end{algorithmic}
\end{algorithm}

The interest, instead, is in exactly how we calculate this
distance. We discussed earlier that we would use native WikiMarkup
tags in order to identify different `species' of text. By doing so, we
could characterise a single revision in terms of the kinds of text
dealt in. By characterising an edit with a series of different edit
difference we also perhaps create the opportunity to consider some as
more valuable than others.

The algorithm that was settled upon left the levenshtein calculator
itself naive of text species -- instead we simply split the text up
and calculate levenstein distance separately. We traverse each string
from beginning to end, using simple regex expressions to identify and
extract different kinds of text, and calculating the levenshtein
distance for each separately. This process is detailed in
algorithm~\ref{dist-calc}. 

%%%%% Distance calculation procedure
\begin{algorithm}
  \caption{Revision pair distance calculation}\label{dist-calc}
  \begin{algorithmic}
    \State $regexes \gets $\{
    \Statex \tab`math1': `$<$math$>$((?!$<${\textbackslash}/math$>$).)*$<${\textbackslash}/math$>${\textbackslash}S',
    \Statex \tab`math2': `\{\{math((?!\}\}).)*\}\}',
    \Statex \tab`bquote': `$<$blockquote$>$((?!$<${\textbackslash}/blockquote$>$).)*$<${\textbackslash}/blockquote$>${\textbackslash}S'
    \Statex  \tab...
    \Statex\}\Comment{Regexes that recognise single Wikimarkup tags}
    \State $reggroups \gets $\{\label{dist-calc-groups}
    \Statex  \tab`maths':(regexes[`math1'], regexes[`math2']),
    \Statex  \tab...
    \Statex \}\Comment{Group of regexes by 'species'}
    \State $distances \gets \emptyset$
    \Function{PairDistance}{$str1,str2$}
    \State $strs \gets [str1, str2]$
    \ForAll {$key, reg \in reggroups$}
    \State $comparestr \gets [``", ``"]$
    \For {$i \gets 0,1$}
    \State $matches \gets reg.matches(strs[i])$
    \ForAll {$m \in matches$}
    \State $match, strs[i] \gets $extractsplit($m.start, m.end, strs[i]$)
    \State $comparestr[i] \gets comparestr[i] + match$
    \EndFor
    \EndFor
    \If {length($comparestr[0]$)$ > 0$ OR length($comparestr[1]$)$ > 0$}
    \State $distances[key] \gets LevDist(comparestr[0], comparestr[1])$ \Comment{See algorithm~\ref{lev-dist}}
    \Else
    \State $distances[key] \gets 0$
    \EndIf
    \EndFor
    \State $distances[$`$norm$'$] \gets LevDist(strs[0], strs[1])$
    \Comment{Process the remainder}
    \State return $distances$
    \EndFunction
  \end{algorithmic}
\end{algorithm}

In practice, this algorithm was much quicker than trying to add an
awareness of text species to the levenshtein calculator itself. We
using regex statements, we can search and split the string relatively
quickly, and use this preprocessing to alleviate the levenshtein
distance calculator of the burden of being aware of the kinds of text
it is dealing with. With this awareness integrated, because
levenshtein distance considers one character at a time, these
operations of flagging and identifying areas of text were inevitably
multiplied many thousands of time in one operation. Instead we were
able to use a fairly simply algorithm for calculating levenshtein
distance, found in algorithm~\ref{lev-dist}.

We will discuss different levenshtein-related algorithms further on
this thesis, but for now we can say that the one we reference here is
fairly basic, but with an optimised space efficiency. We see that we
don't hold a whole matrix for the two strings, only the current and
previous row. We may also describe the PairDistance overall as a
divide-and-conquer algorithm. It improves the space complexity of the
algorithm a little, and allows us the employ parallel or threaded
processing in order to improve efficiency of computation.

\begin{algorithm}
  \caption{Levenshtein distance calculator}\label{lev-dist}
  \begin{algorithmic}
    \Function{LevDist}{$str1, str2$}
    \State $s1len \gets $length($str1$)
    \State $s2len \gets $length($str2$)
    \State $column \gets [0_{1}, 0_{2}, \ldots, 0_{s1len}]$
    \For{$x \gets 1,s2len$}
    \State $col[x] \gets x$
    \EndFor
    \For{$p \gets 1,s1len$}
    \State $column[0] \gets p$
    \State $r \gets p-1$
    \For{$q \gets 1,s2len$}
    \State $oldnum \gets column[q]$
    \State $column[q] \gets min(col[q]+1, col[q-1] + 1, r + str1[p-1] \neq str2[q-1])$
    \EndFor
    \EndFor
    \State return $col[s1len]$
    \EndFunction
  \end{algorithmic}
\end{algorithm}

Our most important algorithm, however, is the trajectory calculation
algorithm. With this algorithm we allow ourselves to automatically
identify some of the context of the revision history, using the same
tools with which we analyse pair distance.

We consider that, although Wikipedia articles have no endpoint as a
rule -- we refer to page~\ref{quote-page} -- we must take the most
recent page in our history sample as a kind of `goal'. We measure
every revision's levenshtein distance from the final version, giving
us an idea of how close to `finished' it is, at least in terms of this
history. We then, taking that levenshtein distance as $\Delta y$, and
the time difference between that revision and it's predecessor as
$\Delta x$, we calculate the following:

\[gfactor(\Delta x,\Delta y) = \left\{ 
\begin{array}{l l l}
  1 & \quad \text{if ${\Delta}x = 0$ and ${\Delta}y < 0$ }\\
  0 & \quad \text{if ${\Delta}x = 0$ and ${\Delta}y >= 0$ }\\
  \frac{arctan({\Delta}y/{\Delta}x)}{\pi}\text{if ${\Delta}x > 0$}
\end{array} \right.\]

This function maps every point on a given arc to a real number in the
range 0 to 1 as shown in figure~\ref{fig:circle-map}· The real number
increases the more acutely the article approaches it's final version:
inserting a lot of text that is eventually deleted results in a small
gradient factor, and including a lot of text that is also in the final
text results in a number closer to one. The time frame of this change
alters the number also -- the sooner an edit is made after the
previous one, the closer the gradient factor will be towards either 0
or 1, depending on the sign of $\Delta y$. (Note that the gradient
factor a negative $\Delta x$ is undefined, as a revision cannot occur
before it's predecessor.)

\begin{figure}
  \centering
  \begin{tikzpicture}[x=1.25cm,y=1.25cm]

    \draw[thick,dashed] (4,0) -- (4,1);
    \draw[thick,dashed] (4,7) -- (4,8);
    \draw[thick,dashed] (7,4) -- (8,4);
    \draw (4,4) circle (3cm);
    \node[fill,circle] at (4,4) (o) {};
    
    \node[fill=white,left=1.5cm of o, minimum height=3cm]{\textit{undefined region}};
    \node at (6,7) {\textit{moving away from final}};
    \node at (6,1) {\textit{moving towards final}};
    
    \draw [thick, ->] (o) -- (7,4) node[sloped, midway, above]{$0.5$};
    \draw [thick, ->] (o) -- (6,2) node[sloped, midway, above]{$0.75$};
    \draw [thick, ->] (o) -- (6,6) node[sloped, midway, below]{$0.25$};
    \draw [thick, ->] (o) -- (4,1) node[midway, right]{$1$};
    \draw [thick, ->] (o) -- (4,7) node[midway, right]{$0$};
  \end{tikzpicture}
  \caption{Mapping of trajectory angle to gradient factor}
  \label{fig:circle-map}
\end{figure}

The significance of this number is two-fold -- it at once describes
the amount of change created in the history in terms of the target of
that history, and the longevity of the article's previous state,
(i.e. it's stability). For an edit to approach 1, it needs not only to
contribute a large proportion of the text's final form, but also to do
so very quickly after the previous revision. Later we will discuss
what we can understand from these values in isolation, and how they
relate to one another as a set. 

Our procedure for collecting these values is described in
algorithm~\ref{traj-calc}.

%%%%% Trace trajectory algorithm
\begin{algorithm}
\caption{Page trajectory calculation}\label{traj-calc}
  \begin{algorithmic}
    \Procedure{TrajectoryCalculation}{$revids, domain$}
    \State $target \gets $database.gettext($revids[-1]$)\Comment{Last revision in list is most recent}
    \For {$i \gets length(revids), 0$}
    \If{trajectory distance not already in database}
    \State $str1 \gets $database.gettext($revids[i]$)
    \State $dist \gets $LevDist($str1, target$)\Comment{See algorithm~\ref{lev-dist}}
    \State $database.inserttrajectoryinsert(dist)$    
    \EndIf
    \EndFor
    \For{$i \gets 0, length(revids)$}
    \State $dist2 \gets $database.gettrajectory($revids[i],domain$)
    \State $dist1 \gets$
        \LineIf{database.gettrajectory($revid[i-1],domain$)}{$i \neq 0$}{$2
          \times disty$}
    \State $time2 \gets $database.gettimestamp($revid[i],domain$)
    \State $time1 \gets $
        \LineIf{database.gettimestamp($revid[i-1],domain$)}{$i \neq 0$}{$timex$} 
    \State ${\Delta}x \gets time2 - time1$
    \State ${\Delta}y \gets dist2 - dist1$
    \If{${\Delta}x > 0$}
    \State $gradient = \frac{arctan({\Delta}y/{\Delta}x)}{\pi}$ 
    \ElsIf{$x = 0$}
    \State $gradient = $\LineIf{1}{$y < 0$}{0}
    \EndIf
    \State database.insertgradient($revid[i],domain,gradient$)
    \EndFor
    \EndProcedure
  \end{algorithmic}
\end{algorithm}
