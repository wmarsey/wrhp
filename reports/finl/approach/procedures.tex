%%WEIGHTING BY MARKUP
\subsection*{Identifying and splitting by text species}
Using the wikimedia markup system, we may easily identify certain
species of text. They are as follows:
\begin{itemize*}
\item Internal links
\item External links
\item Images
\item Files
\item Musical scores
\item Math-formatted text (similar to Latex math environment)
\item Section headings (of differing levels)
\item `Citation Needed' tags
\item `As of' tags (used for identification of age-sensitive
  information)
\item Block quotes
\item Tables
\end{itemize*}
\clearpage
We then group them logically, as follows:
\begin{description*}
  \item[Equation]\hfill\\
    Math-formatted text
  \item[Source validation]\hfill\\
    Block quotes, Citations, `Citation Needed' tags, `As of' tags
  \item[Links]\hfill\\
    Internal Links, External Links
  \item[Structural]\hfill\\ 
    Section headings, Tables\\
    It was found in 2005 that this, if anything, was the clearest
    difference between Wikipedia and commercial
    encyclopedias,\cite{Giles2005} supporting previous
    conjecture.\cite{Denning2005} The regexes in this group provide a
    simple way of noticing changes that affect structure.
\end{description*}

We may split the string along the boundaries of these markup tags, and
send levenshtein distance calculator sperate strings, as shown in
figure~\ref{fig:split-diff}.

With these text species grouped, we m. By characterising an edit with
a series of different edit difference we also perhaps create the
opportunity to consider some as more valuable than
others.\label{multiprocessing-bit}

The algorithm that was settled upon left the levenshtein calculator
itself naive of text species -- instead we simply split the text up
and calculate levenstein distance separately. We traverse each string
from beginning to end, using simple regex expressions to identify and
extract different kinds of text, and calculating the levenshtein
distance for each separately. This process is detailed in
algorithm~\ref{dist-calc}. 

%%%%% Distance calculation procedure
\begin{algorithm}
  \caption{Revision pair distance calculation}\label{dist-calc}
  \begin{algorithmic}
    \State $regexes \gets $\{
    \Statex \tab`math1': `$<$math$>$((?!$<${\textbackslash}/math$>$).)*$<${\textbackslash}/math$>${\textbackslash}S',
    \Statex \tab`math2': `\{\{math((?!\}\}).)*\}\}',
    \Statex \tab`bquote': `$<$blockquote$>$((?!$<${\textbackslash}/blockquote$>$).)*$<${\textbackslash}/blockquote$>${\textbackslash}S'
    \Statex  \tab...
    \Statex\}\Comment{Regexes that recognise single Wikimarkup tags}
    \State $reggroups \gets $\{\label{dist-calc-groups}
    \Statex  \tab`maths':(regexes[`math1'], regexes[`math2']),
    \Statex  \tab...
    \Statex \}\Comment{Group of regexes by 'species'}
    \State $distances \gets \emptyset$
    \Function{PairDistance}{$str1,str2$}
    \State $strs \gets [str1, str2]$
    \ForAll {$key, reg \in reggroups$}
    \State $comparestr \gets [``", ``"]$
    \For {$i \gets 0,1$}
    \State $matches \gets reg.matches(strs[i])$
    \ForAll {$m \in matches$}
    \State $match, strs[i] \gets $extractsplit($m.start, m.end, strs[i]$)
    \State $comparestr[i] \gets comparestr[i] + match$
    \EndFor
    \EndFor
    \If {length($comparestr[0]$)$ > 0$ OR length($comparestr[1]$)$ > 0$}
    \State $distances[key] \gets LevDist(comparestr[0], comparestr[1])$ \Comment{See algorithm~\ref{lev-dist}}\label{mprocess-spawn}
    \Else
    \State $distances[key] \gets 0$
    \EndIf
    \EndFor
    \State $distances[$`$norm$'$] \gets LevDist(strs[0], strs[1])$
    \Comment{Process the remainder}
    \State return $distances$\label{mprocess-return}
    \EndFunction
  \end{algorithmic}
\end{algorithm}

In practice, this algorithm was much quicker than trying to add an
awareness of text species to the levenshtein calculator itself. We
using regex statements, we can search and split the string relatively
quickly, and use this preprocessing to alleviate the levenshtein
distance calculator of the burden of being aware of the kinds of text
it is dealing with. With this awareness integrated, because
levenshtein distance considers one character at a time, these
operations of flagging and identifying areas of text were inevitably
multiplied many thousands of time in one operation. Instead we were
able to use a fairly simply algorithm for calculating levenshtein
distance, found in algorithm~\ref{lev-dist}.

\begin{figure}[p]
  \centering

  \begin{tikzpicture}[
      block1/.style={
        text width=3cm, 
        minimum height=2cm,
        align=center,
        anchor=north,
      },
      block2/.style={
        text width=3cm, 
        minimum height=1.5cm,
        align=center,
        anchor=north,
      },
      font=\small
    ]
    \node (a) [block1] {We can use a [[spork]] to eat spaghetti.};

    \node (b) [block1, right=1cm of a]{We can use [[forks]] to eat lovely
      spaghetti.};

    \node (c) [block1,below=1cm of a]{We can use a \hilight{[[spork]]} to eat
      spaghetti.};

    \node (d) [block1,below=1cm of b]{We can use \hilight{[[forks]]} to eat lovely
      spaghetti.};

    \node (e) [block2,below left=1.5cm and 1.3cm of c] {\hilight{[[spork]]}};
    
    \node (f) [block2,right=1cm of e] {We can use a\ \ to eat spaghetti.};

    \node (g) [block2,right=1cm of f] {\hilight{[[forks]]}};
    
    \node (h) [block2,right=1cm of g] {We can use\ \ to eat lovely spaghetti.};

    \node (i) [block2,below=1cm of e] {\hilight{[[spork]]}}; 
        
    \node (j) [block2,below=1cm of f] {\hilight{[[forks]]}}; 
    
    \node (k) [block2,below =1cm of g] {We can use a\ \ to eat
      spaghetti.};

    \node (l) [block2,below =1cm of h] {We can use\ \ to eat lovely spaghetti.};

    \node (m) [block2,below= 1cm of j] {$ed_{links} = 3$};

    \node (n) [block2,below =1cm of k] {$ed_{normal} = 9$};

    \draw [->] (a) -- (c);
    \draw [->] (b) -- (d);
    \draw [->] (c) -- (e);
    \draw [->] (c) -- (f);
    \draw [->] (d) -- (g);
    \draw [->] (d) -- (h);
    \draw [->] (e) -- (i);
    \draw [->] (g) -- (j);
    \draw [->] (f) -- (k);
    \draw [->] (h) -- (l);
    \draw [->] (i) -- (m);
    \draw [->] (j) -- (m);
    \draw [->] (k) -- (n);
    \draw [->] (l) -- (n);

  \end{tikzpicture}
  \caption{Diagram identification of link text, splitting text into
    normal and link segments, and performing separate levenshtein
    distance calculations}
\label{fig:split-diff}
\end{figure}

%%%%% Pair comparison algorithm
\begin{algorithm}[p]
\caption{Pair comparison}\label{pair-comp}
  \begin{algorithmic}
    \Procedure{PairComparison}{$revids$}
    \For {$ i \gets 0, $length($revids$)}
    \If{pair distance not already in database}
    \State $str1 \gets $\LineIf{``"}{$i=0$}{database.gettext($revs[i-1]$)}
    \State $str2 \gets $database.gettext($revs[i]$)
    \State $dist \gets $PairDistance($str1, str2$)\Comment{See algorithm~\ref{dist-calc}}
    \State databaseinsert.pairdistanceinsert($dist$)  
    \EndIf
    \EndFor
  \EndProcedure
  \end{algorithmic}
\end{algorithm}

We will discuss different levenshtein-related algorithms further on
this thesis, but for now we can say that the one we reference here is
fairly basic, but with an optimised space efficiency. We see that we
don't hold a whole matrix for the two strings, only the current and
previous row. We may also describe the PairDistance overall as a
divide-and-conquer algorithm. It improves the space complexity of the
algorithm a little, and allows us the employ parallel or threaded
processing in order to improve efficiency of computation.

\begin{algorithm}
  \begin{algorithmic}
    \Function{LevDist}{$str1, str2$}
    \State $s1len \gets $length($str1$)
    \State $s2len \gets $length($str2$)
    \State $column \gets [0_{1}, 0_{2}, \ldots, 0_{s1len}]$
    \For{$x \gets 1,s2len$}
    \State $col[x] \gets x$
    \EndFor
    \For{$p \gets 1,s1len$}
    \State $column[0] \gets p$
    \State $r \gets p-1$
    \For{$q \gets 1,s2len$}
    \State $oldnum \gets column[q]$
    \State $column[q] \gets min(col[q]+1, col[q-1] + 1, r + str1[p-1] \neq str2[q-1])$
    \EndFor
    \EndFor
    \State return $col[s1len]$
    \EndFunction
  \end{algorithmic}
  \caption{Levenshtein distance calculator}
  \label{lev-dist}
\end{algorithm}

Our procedure for collecting these values is described in
algorithm~\ref{traj-calc}.

%%%%% Trace trajectory algorithm
\begin{algorithm}
  \caption{Page trajectory calculation}
  \label{traj-calc}
  \begin{algorithmic}
    \Procedure{TrajectoryCalculation}{$revids, domain$}
    \State $target \gets $database.gettext($revids[-1]$)\Comment{Last revision in list is most recent}
    \For {$i \gets length(revids), 0$}
    \If{trajectory distance not already in database}
    \State $str1 \gets $database.gettext($revids[i]$)
    \State $dist \gets $LevDist($str1, target$)\Comment{See algorithm~\ref{lev-dist}}
    \State $database.inserttrajectoryinsert(dist)$    
    \EndIf
    \EndFor
    \For{$i \gets 0, length(revids)$}
    \State $dist2 \gets $database.gettrajectory($revids[i],domain$)
    \State $dist1 \gets$
        \LineIf{database.gettrajectory($revid[i-1],domain$)}{$i \neq 0$}{$2 \times disty$}
    \State $time2 \gets $database.gettimestamp($revid[i],domain$)
    \State $time1 \gets $
        \LineIf{database.gettimestamp($revid[i-1],domain$)}{$i \neq 0$}{$timex$} 
    \State ${\Delta}x \gets time2 - time1$
    \State ${\Delta}y \gets dist2 - dist1$
    \If{${\Delta}x > 0$}
    \State $gradient = \frac{arctan({\Delta}y/{\Delta}x)}{\pi}$ 
    \ElsIf{$x = 0$}
    \State $gradient = $\LineIf{1}{$y < 0$}{0}
    \EndIf
    \State database.insertgradient($revid[i],domain,gradient$)
    \EndFor
    \EndProcedure
  \end{algorithmic}
\end{algorithm}
