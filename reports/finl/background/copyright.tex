\section{Copyright}
\label{sec:copyright}
We first look to existing copyright legislation on joint
authorship. In particularly, we look to Intellectual property law --
laws that govern `creations of the mind'. It may be argued that
Wikipedia writers are engaged in `creations of the mind', but are
rather analogous to copy writers. In any case the rules are the same
-- the content, and the construction of the content, is the part that
matters, not the ideas themselves.\cite{i-property}

We keep in mind that there is no world standard legislature regarding
intellectual property, though Wikipedia is an international
platform. (They are governed by United States copyright law, but
endeavour to respect the copyright law of all countries, even if these
do not have official copyright relations with the United
States.\cite{wiki-letter})

Firstly, we review the general definitions of a joint work. According
to American legislature:

\begin{quote}
  When two or more authors prepare a work with the intent to combine
  their contributions into inseparable or interdependent parts, the
  work is considered joint work and the authors are considered joint
  copyright owners.\cite{what-joint-authorship}
\end{quote}

We can apply this confidently to Wikipedia. Editors of course know
their efforts will be combined. The UK legislature is similar,
defining a joint work thus:

\begin{quote}
  Where two or more people have created a single work protected by
  copyright and the contribution of each author is not distinct from
  that of the other(s).\cite{joint-authorship}
\end{quote}

Though, joint authorship has no direct implications with regard to
licensing. As far as most law is concerned, in joint-authored works,
terms of usage, and therefore distribution of funds, must be agreed
separately in each case:

\begin{quote}
  ownership of copyright can be transferred, so where something is
  produced that has involved contributions from more than one person,
  it would be possible for copyright in all the material to be owned
  by one person as a result of appropriate transfers. Indeed,
  collaborators can agree in advance that copyright in what is to be
  produced should be owned by a single person or body. 

  ... 

  However, alternative solutions that might be equally helpful could
  involve all parties agreeing licensing arrangements in
  advance.\cite{joint-authorship}
\end{quote}

In the former case described above, the single `owner' will sometimes
be assigned the rights of all parties on agreement that they divvy up
any profits according to the contract of that assignment. These
profits are the parties' `royalty rates'. In the latter case, the
division of these proceeds by a pre-agreed royalty rate is a given.

How these royalty rates are calculated, however, is left to the
discretion of the individuals involved. Industry literature defines
the different methods for deciding upon these different rates:
comparison with previous similar deals done by others, alignment with
industry or internal practice, and calculation. The first method
doesn't necessary reflect differences in cases, the second can be
difficult when some parties have limited bargaining power. The third
can be the most rational, though is often highly
complex.\cite{simplemethod}

This latter-most method seems to be the most applicable model to
address the current situation. However, current models that have the
royalty share based upon financial investment are
inappropriate.\cite{simplemethod} To achieve our goal of dividing
share automatically, we are faced with defining a different kind of
economy.

Writers are most generally hired by the hour / day, or commissioned by
word count.\cite{copywriter-rates} We have no data about how long an
editor may spend on an article, and when there is a high likelihood
that an edit will be refined, or partially deleted or rewritten by
another, word count seems to be fairly inappropriate also.

In the context of the organic, open-source development of a Wikipedia
article (Wikipedia's answer to who is responsible for articles? ``You
are!''\cite{wiki-you-are}), it seems most natural to pursue measures
of how an edit may have positively affected the growth of an article
--- it's consensus.

We can define a positive edit in two ways. A positive edit may
be one that upholds the values of a good Wikipedia article, i.e. they
are:

\begin{quote}
  written very well, contain factually accurate and verifiable
  information, are broad in coverage, neutral in point of view,
  stable, and illustrated, where possible, by relevant images with
  suitable copyright licenses.\cite{wiki-good}
\end{quote}

Or, they help the article along towards it's final version. In this we
may approach the state above according to this heuristic:

\begin{quote}
  An article will automatically approach the `golden' standard of
  Wikipedia.
\end{quote}

According to this heuristic, if a majority of editors approach an
article with the above standards in mind, any edits that do not meet
these standards will be undone or changed so as to approach these
standards. And the heuristic seems plausible. Indeed, Wikipedia even
has it's own Wikipedia-local definition of the word consensus: ``the
primary way decisions are made on Wikipedia, and it is accepted as the
best method to achieve \textit{our goals}.''\cite{wiki-consensus}

According to these two techniques, we define an economy based on
adherence to native Wikipedia standards of quality, and divide share
based upon this. 
