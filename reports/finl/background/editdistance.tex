\section{Edit difference algorithms}
To measure difference between different text revisions, we refer to
edit distance. The edit distance between two texts, first defined by
Vladimir Levenshtein in the 1960s,\cite{Levenshtein1966} can be
defined as the minimum amount of insert, delete and substitutions
operations needed to transform one text into another. We find a
trivial illustration of the edit distance primitives in
figure~\ref{fig:fork-spork}

\begin{figure}
  \centering
  
  \begin{tikzpicture}[node distance =3pt and 0.5cm,anchor=center]
    
    \matrix[table] (mat11) { |[fill=greeni]| &
      |[fill=yellowi]|F & O & R & K & |[fill=redi]|S
      \\ |[fill=greeni]|S & |[fill=yellowi]|P & O & R & K &
      |[fill=redi]| \\ };

    \CellText{11-1-1}{string 1:}; \CellText{11-2-1}{string
      2:};

    \SlText{11-1-1}{Insert} \SlText{11-1-2}{Swap}
    \SlText{11-1-6}{Delete}
    
  \end{tikzpicture}

  \vspace{3 mm}

  forks $\rightarrow$ spork, edit distance: 3
  
  \caption{An edit distance example using all three basic edit
    operations}
  \label{fig:fork-spork}
\end{figure}

\begin{figure}
  \centering
  for the function $\mbox{lev}_{a,b}(|a|,|b|)$:\\
  $$\mbox{lev}_{a,b}(i,j) = 
  \left\{
  \begin{array}{ll}
    \mbox{max}(i,j) & \mbox{if }min(i,j) = 0\\
    \mbox{min}\left\{
    \begin{array}{lll}
      lev_{a,b}(i-1,j)+1\\
      lev_{a,b}(i,j-1)+1\\
      lev_{a,b}(i-1,j-1)+1_{(a_i{\neq}b_j)}
    \end{array}
    \right.
    & else 
  \end{array}
  \right.$$
  when $a_i = b_j$, $1_{(a_i{\neq}b_j)} = 1$\\
  when  $a_i \neq b_j$, $1_{(a_i{\neq}b_j)} = 0$
  \caption{The definition of Levenshtein edit distance.}
  \label{fig:levdef}
\end{figure}

Levenshtein's own characterisation of this distance is given in
figure~\ref{fig:levdef}. It defines that the distance between two
strings is characterised the minimum distance between three different
pair-combinations of its substrings. A `text-book' implementation of
this algorithm can be represented by the pseudo-code in
figure~\ref{ref:levenshtein-dynamic}. We present the
dynamic-programming-style algorithm here, and will generally be
working with dynamic programming implementations throughout the study.

\begin{figure}
  REDO AS ALGORTHM?!
  \centering
  \begin{lstlisting}
    ed(x,y):
    #base cases
    if |x| = 0: return |y|
    if |y| = 0: return |x|    

    #table initialisation
    d is a table [0..|x|][0..|y|]
    for i = 1 to |m|:
    d[i,0] = i
    for j = 0 to |y|:
    d[0,j] = j           
    
    #dynamic computation
    for j = 1 to |y|:
    for i = 1 to |x|:
    c = [(x[i] == y[j]) ? O else 1]
    ins = d[i-1,j] + 1
    dlt =d[i,j-1] + 1
    kp_swp = d[i-1,j-1] + c
    d[i,j] = min(ins, dlt, kp-swp)
    
    #return last computed number
    return d[|x|,|y|]
  \end{lstlisting}
  \caption{Basic dynamic implementation of Levenshtein distance}
  \label{fig:levenshtein-dynamic}
\end{figure}

We can see that on comparing strings $x$ and $y$, a
$|x|$ by $|y|$ table is created, and then filled with values. For this
reason both the time and space complexity of the algorithm is $\theta
(|x||y|)$.

Reducing the space needed for this computation is relatively easy, and
can be done in a few different ways. One way is to simply disregard
parts of the table already computed. We can see that, on each
computation of $d[i,j]$ (as it appears above), we require only a small
part of the matrix: $d[i-1,j-1]$, $d[i-1,j]$ and $d[i,j-1]$. At any
iteration $i$, where is great than $1$, we may disregard rows $0 \dots
(i-2)$ inclusive. We eventually implement this version, implementing
the algorithm~\ref{lev-dist}, on page~\pageref{lev-dist}.

There are more complicated techniques that allow us to also disregard
unneccesary computation --- a few implementations employ strategies
that allow them to trace the table space diagonally, rather than
iteratively, achieving a time complexities as low as $O(ed(x, y)^2)$
(though they sacrifice some accuracy).\cite{Chang1992} Others
harnesses the speed of bit operations to achieve a time complexity of
$O(nm/w)$ or $O(nm log {\Sigma}/w)$ time where $w$ the bit-word size
of the machine, and $\Sigma$ is the alphabet
size.\cite{Myers1999}\cite{Hyyro2003}

In this project, however, it was suffice to simply reduce the space
needed for the computation, as the texts were relatively small, and
speed not an issue. However, we were able to speed up the program
somewhat by implementing some simple multiprocessing. Both these
procedures, and their limitations, are discussed in more detail on
page~\pageref{multiprocessing-bit}.

%% \subsection*{Varieties of edit distance}
%% Modifications can be made to the nature of the distance itself, in
%% order to adapt the measure a variety of different and specific
%% needs. Here is a brief overview of the main groups these extensions
%% fall into:

%% \begin{itemize}
%% \item \textbf{Hamming distance.} This allows for substitutions only,
%%   comparing same-length strings, such
%%   that:\\ $ed_{hamming}(\text{``abc''},\text{``abd''})
%%   =1$,\\ $ed_{hamming}(\text{``abc''},\text{``bcd''}) = 3$,\\ and
%%   $ed_{hamming}(\text{``abc''},\text{``ab''})$ is
%%   undefined.\cite{Hamming1950}
%% \item \textbf{Reversals.} The Damerau-Levenshtein distance defines an
%%   `swap' operation, which is the reversal of two adjacent
%%   characters. It is particularly suited to spell-checking, and for
%%   analysing DNA-sequence variations. In this
%%   case:\\ $ed_{damerau}(\text{``ab''},\text{``ba''}) = 1$
%% \item \textbf{Block distance.} This allows for displacements of entire
%%   blocks to count as one operation. For
%%   example:\\ $ed_{block}(\text{``abcde''},\text{``cdeax''})= 2$ \\ One
%%   move of the block `cde', one substitution of `b' for
%%   `x'.\cite{Tichy1984}
%% \item \textbf{\textit{q}-grams distance.} \textit{q}-grams are simply
%%   sub-strings, and this measure describes the similarity of two
%%   strings in terms of \textit{q}-grams they
%%   share.\cite{Ukkonen1992}\\ $ed_{q-gram}(x,y)=\sum\limits_{v\in\Sigma
%%     ^q}|G(x)[v]-G(y)[v]|$\\ where $G(x)[v]$ returns the number of
%%   occurrences of \textit{q}-gram v in string x, and $\Sigma ^q$ is all
%%   the possible \textit{q}-grams in the alphabet (capped by string
%%   length). $|G(x)[v]-G(y)[v]|$ a large positive number every time a
%%   \textit{q}-gram appears a large amount of times in one string, but
%%   not the other; it returns 0 if the substring apears the same number
%%   of times. So, the whole function measures this difference for all
%%   possible substrings, and sums them, returning a high number for
%%   difference, and a low number for similarity.
%% \end{itemize}


\subsection*{Optimal alignment}
Another part of the problem of working out optimal edit distance is
finding the `optimal alignment' --- the measures are closely
related. We displace and arrange the characters of a string such that
the set of operations to transform each character into its counterpart
is minimal. For example, in figure \ref{fig:fork-spork}, the alignment
of the two strings ``fork'' and ``spork'' was:

\begin{center}
  \begin{tabular}{cccccc}
    s & p & o & r & k & -\\
    - & f & o & r & k & s 
  \end{tabular}
\end{center}

However it could also conceivably have been:

\begin{center}
  \begin{tabular}{ccccccccccccccccc}
    s & p & o & r & k & - & & or even & & - & s & p & o & - & r & k - &\\
    f & - & o & r & k & s & &         & & f & - & o & - & r & k & - & s    
  \end{tabular}
\end{center}\label{fig:sub-opt}

Here, the left-hand version results in an equivalent Levenshtein
distance, but we can see how the distance for the right-hand example
would be sub-optimal, requiring 8 edit operations.

\begin{figure}[h]
  \centering   
  \begin{tikzpicture}[node distance =3pt and 0.5cm,anchor=center]
    \matrix[table] (mat11) {|[fill=greeni]|  & |[fill=redi]|S & |[fill=yellowi]|P & |[fill=redi]|O & |[fill=greeni]|  & |[fill=redi]|R & K & |[fill=greeni]|\\
      |[fill=greeni]|F & |[fill=redi]|  & |[fill=yellowi]|O & |[fill=redi]|  & |[fill=greeni]|R & |[fill=redi]|  & K & |[fill=greeni]|S\\};
    
    \CellText{11-1-1}{string 1:}; \CellText{11-2-1}{string
      2:};

    \SlText{11-1-1}{Insert}
    \SlText{11-1-2}{Delete}
    \SlText{11-1-3}{Swap}
    \SlText{11-1-4}{Delete}
    \SlText{11-1-5}{Insert}
    \SlText{11-1-6}{Delete}
    \SlText{11-1-8}{Insert}
  \end{tikzpicture}\\
  \vspace{3 mm}
  spork $\rightarrow$ forks, edit distance: 7
  \caption{An sub-optimal edit distance example}
  \label{fig:fork-spork-subopt}
\end{figure}

The Smith-Waterman algorithm calculates optimal alignment by
populating two tables -- one like that in the pseudocode above, and
also as a table of directions.\cite{smithwaterman} These directions describe paths from
one corner of the table space to the other; the shape of this path
can define how to align the two strings.\cite{Smith1981}

This path may also be read as an edit operation. An arrow at the
position $[i,j]$ in the table defines edit operations for $x[i]$
and/or $y[j]$, as described in figure
\ref{fig:smith-waterman-traceback}.

\begin{figure}[h]
  \centering 
  \begin{itemize}
  \item \textbf{$\nwarrow$ at $[i,j]$, if $x[i] \neq y[j]$} \\ Denotes a
    'swap' between $x[i]$ and $y[j]$ (if $x[i] = y[j]$ then it denotes
    the lack of an operation).
  \item \textbf{$\uparrow$ at $[i,j]$}\\Denotes the deletion of $x[i]$
  \item \textbf{$\leftarrow$ at $[i,j]$}\\Denotes the insertion of
    $y[j]$
  \end{itemize}
  \vspace{10mm}

  $\left\{
  \begin{array}{ccccccc}
    & & S & P & O & R & K \\ & \color{red}{0} & \color{red}{0} & 0 & 0
    & 0 & 0 \\ F & 0 & \nwarrow & \color{red}{\nwarrow} & \nwarrow &
    \nwarrow & \nwarrow \\ O & 0 & \uparrow & \nwarrow &
    \color{red}{\nwarrow} & \downarrow & \leftarrow \\ R & 0 &
    \uparrow & \uparrow & \uparrow & \color{red}{\nwarrow} &
    \leftarrow \\ K & 0 & \uparrow & \uparrow & \uparrow & \uparrow &
    \color{red}{\nwarrow} \\ S & 0 & \nwarrow & \uparrow & \uparrow &
    \uparrow & \color{red}{\uparrow} \\
  \end{array}\right\} $\\
  (If the arrow reaches an edge before the left-hand corner, we trace
  along that edge, reading each shift as an arrow in the direction of
  the trace.)
  \caption{Diagram showing Smith-Waterman traceback path (in red) on
    the edit operation forks $\rightarrow$ spork}
  \label{fig:smith-waterman-traceback}
\end{figure}

Our early implementations of the Levenshtein algorithm implemented
various ways of computing this alignment data, including creating a
separate matrix as here, and implementing a struct which would accrue
this path data as it was passed along in the $min(a,b,c)$ function of
the Levenshtein algorithm. However, these implementations were very
slow, and we decided that evidencing alignment was not a priority. 

We did, however, find that splitting the string caused some
sub-optimal alignment, and discrepancies in plain levenshtein vs split
levenshtein distance. We discuss this in detail on
page~\pageref{split-distance-eval}.

%% \subsection*{Delta encoding}
%% Finally, we may also look into Delta encoding algorithms. These
%% describe ways of compressing the storage of a document's history --- a
%% format in which only the differences between each text is stored, not
%% the entire version. These algorithms are of the same family of
%% algorithms discussed above. In fact, we find that one of the fastest
%% known algorithms,\footnote{According to Hunt's 1998
%%   study\cite{Hunt1998}} \textit{VDelta}, is a refinement of the block
%% distance algorithm mentioned above. For a given version $n$ of a
%% document $doc$ is defined as:

%% $$v_n = v_0 \cup {\Delta}(v_0,v_1) \cup {\Delta}(v_1,v_2) \cup \dots
%% \cup {\Delta}(v_{n-1},v_n) $$

%% where ${\Delta}(v_i,v_j)$ is the difference between version $i$ and
%% version $j$ of the document, and the union operation $\cup$ combines
%% each version in a manner particular to the $\Delta$ data-type. Storing
%% data in this way can be very efficient, resulting in a compression
%% factors of five or ten on typical data.\cite{Macdonald2000} It may
%% also be relatively easy to maintain in our case, due to the linear
%% nature of Wikipedia revision histories.
