\chapter{Fetching the Data}

% Define block styles
\tikzstyle{decision} = [rectangle, 
  draw, 
  fill=blue!20, 
  text width=5em, 
  text badly centered, 
  node distance=3cm,
  inner sep=0,
  outer sep=0,
  inner sep=0pt, 
  font=\footnotesize,
rounded corners,
execute at begin node=\setlength{\baselineskip}{1em}]
\tikzstyle{block} = [rectangle, 
  draw, 
  fill=green!20, 
  inner sep=0,
  outer sep=0,
  text width=6em, 
%  text width=5em,
  text centered,
  font=\footnotesize,
  execute at begin node=\setlength{\baselineskip}{1em}] 
%rounded corners, 
%minimum height=4em]
\tikzstyle{line} = [draw, -latex']
\tikzstyle{cloud} = [draw, 
  ellipse,
  fill=red!20,
  inner sep=0,
  outer sep=0,
  node distance=3cm,
  minimum height=2em,
  font=\footnotesize,
  execute at begin node=\setlength{\baselineskip}{1em}]
    
\begin{figure}
  \begin{tikzpicture}[node distance = 1cm]
    % Place nodes

    \node [cloud] (start) {Start.};
    
    \node [block, right of=start, node distance=2cm] (init) {Initialise
      visited pages, cached pages and corrupt pages arrays.};
    
    \node [decision, right of=init, node distance=2cm] (dompick) {Do we
      have a domain?};

    \node [block, below of=dompick] (domrandom) {Pick a random domain};

    \node [decision, right of=domrandom, node distance=2cm] (pagepick)
          {Do we have a pageid?};
          
          \node [block, below of=pagepick] (pagerandom) {Fetch random pageid
            from current domain};
          
          \node [decision, right of=pagerandom, node distance=2cm] (dbcheck) {Does the current
            ID and domain exist in the database already?};

          \node [block, below of=dbcheck] (fetchrecent) {Fetch most recent
            revision ID};
          
          \node [decision, left of=dbcheck, node distance=2cm] (corruptcheck) {Is the fetched
            data corrupt?};
          
          \node [decision, left of=corruptcheck, node distance=2cm]
          (corruptcheckb) {Is the parentid intact?};
          
          \node [block, right of=corruptcheckb, node distance=3cm]
          (appendcorrupt) {append to corrupt pages array};
          
          \node [block, below of=corruptcheckb, node distance=3cm]
          (appenddisc) {Append to caches pages array.};
          
          \node [block, below of=appenddisc] (idfetch) {Update revid using
            parent ID from held data};
          
          \node [decision, below of=idfetch] (enough) {Do we have 50 pages
            in the cached pages array?};

          \node [block, right of=enough, node distance=3cm] (store) {Store all pages in the cached pages array.};
          
          \node [decision, below of=enough] (parentcheck) {Is parent 0, or
            does exist in the visited array};
          
          \node [decision, below of=parentcheck] (enoughb) {Do we have over X
            revisions in the visited array?};
          
          \node [block, below of=parentcheck] (storeb) {Store all pages in the cached pages array.};

          \node [decision, below of=storeb] (corruptdeal) {Do we have any
            pages in the corrupt array?};

          \node [block, right of=corruptdeal, node distance=3cm] (corruptproc) {Change
            pointers in database in order to 'skip over' corrupt pages.};

          \node [decision, right of=parentcheck] (discard) {Discard page};

          \node [decision, below of=corruptdeal] (continue) {Continue?};

          \node [cloud, below of=continue] (end) {End.};

          % Draw edges
          \path [line] (start) -- (init);

          \path [line] (init) -- (dompick);

          \path [line] (dompick) -- node [near start] {yes} (pagepick);

          \path [line] (dompick) -- node [near start] {no} (domrandom);

          \path [line] (domrandom) -- (pagepick);

          \path [line] (pagepick) -- node [near start] {yes} (fetchrecent);

          \path [line] (pagepick) -- node [near start] {no} (pagerandom);

          \path [line] (pagerandom) -- (fetchrecent);

          \path [line] (fetchrecent) -- (dbcheck);

          \path [line] (dbcheck) -- node [near start] {yes} (idfetch);

          \path [line] (dbcheck) -- node [near start] {no} (corruptcheck);
          
          \path [line] (corruptcheck) -- node [near start] {yes} (corruptcheckb);

          \path [line] (corruptcheck) -- node [near start] {no} (appenddisc);

          \path [line] (appenddisc) -- (idfetch);

          \path [line] (corruptcheckb) -- node [near start] {yes} (continue);

          \path [line] (corruptcheckb) -- node [near start] {no} (appendcorrupt);

          \path [line] (appendcorrupt) -- (idfetch);

          \path [line] (idfetch) -- (enough);

          \path [line] (enough) -- node [near start] {yes} (store);

          \path [line] (enough) -- node [near start] {no} (parentcheck);

          \path [line] (store) -- (parentcheck);

          \path [line] (parentcheck) -- node [near start] {yes} (enoughb);

          \path [line] (parentcheck) -- node [near start] {no} (dbcheck);

          \path [line] (enoughb) -- node [near start] {yes} (storeb);

          \path [line] (enoughb) -- node [near start] {no} (discard);

          \path [line] (discard) -- (continue);

          \path [line] (storeb) -- (corruptdeal);

          \path [line] (corruptdeal) -- node [near start] {yes} (corruptproc);

          \path [line] (corruptdeal) -- node [near start] {no} (continue);

          \path [line] (corruptproc) -- (continue);

          \path [line] (continue) -- node [near start] {yes} (init);

          \path [line] (continue) -- node [near start] {no} (end);

  \end{tikzpicture}
\end{figure}


\end{document}

\section{The Wikipedia API}
WikiMedia provides well-documented API for sites built using their
software, detailed at LINK. It provides the 

\section{The WikiRevision package}
