\section{Summary of Thesis Achievements}
We have looked into the possibility towards the idea of quality in
Wikipedia. 

We saw that many MANY studies use many seemingly arbitrary means to
find GOOD in a wikipedia.

We wanted something more generic. There were studies that used less
arbitrary means. CITE crowdsourced WikiVandalism corpus. 

So we considered looking at the current point of an article to be a
kind of culmination point. This seemed appropriate for Wikipedia's
philosophy. See quote. But also for the nature of knowledge itself --
we learn new things, we incorporate them. A conclusion can keep being
added to.

ALSO MAYBE just maYBE an article has value in the past. but we're not
there.

In the end we are seemingly brought to the conclusion that naive
textual analysis does not affect survival rate of text -- there are
simply too many contextual factors governing it. From our failure to
find correlation between our gradient factor and our other dat (see
section~\ref{mlisbad}), we may begin to conclude that text type isn't
innately connected to 

So, with our automatically we are unable to predict the shape of the
trajectory graph until the entire history is fetched. Does that mean
this is a failure? 

There is a hope we can predict success on edit count, `seniority' of
editor. We could perhaps analyse the edit with awareness, looking for
corroboration online with the information contained in the
text. Way beyond the scope of this project, but possible.

However, we can't predict the future. And if we can predict the
edit-power of an editor, we can't necessarily predict his opinion. Or
even his mood at the time of editing. Refer to Rupert Sheldrake.
