\section{Applications}
The software, as it stands, provides a fairly open and malleable
source for examining the histories of Wikipedia articles. As a
history-visualisation tool, it is different to existing tools (the
most comparable is IBM's blah, though it is more concerned about the
history of each section of a wikipedia article) in that allows for
more flexibility in analysing pages. We have shown how it may also be
leveraged to combine plots of different pages, to allow for greater
context.

The modularity of the approach also allows for more radical changes to
the software. The project could be easily adapted to handle different
data, with little change. To handle GitHub document history, merely
changing the database wrapper and API wrapper classes could be enough
to begin analysing the data there. The anlaysing class asks for text
content based on a two unique identifier -- the file ID and the
revision ID. The trajectory graph, for instance, could be plotted
without trouble.

IF HAVE TIME ACTUALLY DO THAT JUST FOR A SECOND, of the analysing
class. Lolz.
