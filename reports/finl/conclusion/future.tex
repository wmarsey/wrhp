\section{Future Work}
\subsubsection*{Updating gracefully}
GUI? Would be nice to give people better insight to stuff. Was working
on GUI to allow people to play with weights, click to launch
diff. Easy but fiddly. Ran out of time.

\subsection*{Awarding restructuring}
\label{restructuring}
It has been found that, even in the most accurate articles, that the
structure of Wikipedia article can be weak.\cite{Giles2005} We made
some efforts to recognise attempts at structural change, see REFERENCE
TO REGEX DESCRIPTION. However, these attempts are limited. We really
need to be able to identify the move of large blocks of text -- not
only inside an article, but to another. Regarding the former, we can
employ various tecniques WHAT ARE THEY THO. With a larger dataset and the
necessary computing power this would just be a matter of
crossreferencing the deletion of largeish chunks of text with
insertions of similar text within a short time frame.

\subsection*{Refinement of gradient factor}
A limitation of gradient factor is it's reliance upon a relatively
arbitrary measure of time. Though the measure of time has been
consistent throughout the study, the measurement does affect the
distribution of possible gradient factorvalues to the extremeties of
the range. It would be recommended in further research to take this
into account. If someone makes an edit, commits it, fixes spelling,
and commits again, does he recieve less reward than if he had
submitted at once? Or more? This is something we wished we would have
had time to look into during the study.

\subsubsection*{Extending data}
Grabbing each page as we went was useful for cleaning and sometimes
preprocessing parts of the data, but the fetches were time
consuming. Also, when we came to look for more general data about
individual users, and tried to find inter-article connections, we had
to do so in the knowledge that our databases represented a very small
selection from the wikipedia's in question.

For further work on could be better off using a wikipedia dump. These
are made monthly and represent the entirety of the site, history
included at the point of export. They are, however, $~800$GB,
compressed.\cite{wiki-dump} A past studies used a hadoop framework
to enable the study of this data. In the study here we saved time by
simply setting up the API access, but perhaps further study could
invest time in setting up a more complete dataset.

\subsubsection*{Further subjects}
As mentioned previously, this project may easily expand beyond the
wikipedia dataset. A git project history, for instance, may be of
interest for further study. We may fairly combine the existing
research with metrics that concern code in particular, such as
Cyclomatic Complexity, which measures code flow complexity according
to its logical operators.\cite{McCabe1976} The main challenge on
platforms would be mining the contextual data that was so easily found
on Wikipedia.

\subsection*{Summary}
With these latter cases we see the effects of things external to the
articles affecting their content, and how we may bring these things to
bear on the calculation of share. 

It is perhaps one of the most interesting parts of Wiki analysis that
evidence such interpersonal acts is so readily available. Due to the
nature of the WikiMedia software, these interactions are as publicly
available as the article's themselves, all utilising the same text
revision framework. 

Though, we are concious that in other forms of collaboration -- for
which the Wiki dataset is standing in for proof-of-concept study --
these interactions are less readily available. In cases of multi-user
source control, these interactions may occur over personal emails, on
private forums, or in the physical space of a shared workspace.

In this way then, the Wikipedia dataset is almost too complete. By
tunnelling all activity through a common system, we are able to
discover all organisation that surround the creation of the
collaborative works. And while we see that these interactions give
much greater context to collaborative contributions, we must
understand that this kind of data is unavailable on some platforms.

Our findings corroborate with existing research on individual user
activity in large-scale collaboration platforms. 

It has been found that the population and dynamics of an online space
may affect individual user behaviour. Jones, Ravid, Rafaeli find that
Newsgroups users publish shorter posts in more populous threads.

INSERT THING ABOUT GOOGLE POPULAR THING

%% In the world of physical collaboration these kind of interpersonal
%% artefacts of cooperation are apparent. With wikipedia we get a clearly
%% traceable stuff.

%% PLOT DISTRIBUTION OF EDIT NUMBER ON EN, DE, IT

%% PLOT DISTRIBUTION OF SPECIES OF TEXT 

%% THE CASE FOR STUDYING NETWORK 

%% --Frederick Chopin + talk
%% --Rupert Sheldrak + talk
%% --Derek Smart + talk
%% --Aluminium + talk

%% \url{https://en.wikipedia.org/wiki/Wikipedia:Harassment#User_space_harassment}
%% \url{https://en.wikipedia.org/wiki/Wikipedia:Wikipedia_is_in_the_real_world}
%% \url{https://en.wikipedia.org/wiki/Wikipedia:Wikipedia_is_anonymous}
%% \url{https://en.wikipedia.org/wiki/Wikipedia:Harassment#Wikihounding}
