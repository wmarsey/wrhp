\section{Future Work}

From this project, two different directions are recommended. We can
either use the existing technique on text-history system or we can
change the current system to make better analyses of the existing
network. For instance, analyse other parts of Wikipedia in order to
understand better the context of each edit, like the talk page. 

\subsubsection*{Extending data}
Current intentions are to approach the data on an article-by-article
basis, grabbing a particular version and tracing its history
backwards. We grab the pages using HTTP requests, via Mediawiki's
inbuilt API. We could, however, download Wikipedia in its
entirety. The entire site is compressed and dumped monthly, and the
dumps are free to download (though they are 800GB
compressed).\cite{wiki-dump}


\subsubsection*{Further subjects}
The project may well extend to subjects beyond Wikipedia. A git
project history, for instance, may be of interest for further
study. We may combine the existing research with metrics that concern
code in particular, such as Cyclomatic Complexity, which measures code
flow complexity according to its logical operators.\cite{McCabe1976}
It would also be interesting to figure out a way of changing our
algorithm in order to regard non-linear revision histories.

%%COULD HAVE DONE RESTRUCTURING
\subsection*{Awarding restructuring}
\label{restructuring}
It has been found that, even in the most accurate articles, that the
structure of Wikipedia article can be weak.\cite{Giles2005} We should
award attempts to reorganize articles, possibly looking out for lange
block displacements. 
