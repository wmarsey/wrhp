\section{Summary}
Wikipedia proved to be a good case study for the exploring how we may
automatically analyse an individual's collaborative stake in a joint,
text based work.

We were able to analyse the greater context of each article, as well
as the content text. It is perhaps one of the most interesting parts
of Wiki analysis that evidence of interaction is so readily
available. Due to the nature of the WikiMedia software, these
interactions are as publicly available as the article's themselves,
all utilising the same text revision framework. Through this we were
able to expand our knowledge of the circumstance of the collaboration,
and by proxy our understanding of each collaborative act.

Our conclusions are ambiguous. As this work stands, building upon the
work of the previous year to achieve an automatic and expandable
framework for analysis, it makes a healthy attempt at describing each
edit in terms of it's input. We may even describe our weighted
levenshtein distance calculator to be a lossy compression algorithm,
reducing the text down to its meaningful parts.

However, our second measure, gradient factor, which describes both the
local and long-term success of a revision did not seem to correlate
well with the above.

We acknowledge that na\"ive string analysis, by definition, excludes
the recognition of quality content in terms of facts, well-formedness,
and so on, so perhaps our measure is incomplete. These qualities
affect the result of a collaborative act, of course, particularly when
contributing to a knowledge repository. But we also acknowledge that
the correlation between edit content and edit success may not have a
simple correlation.

For all the studies that sought out to analyse objective quality in
articles, and edits, there are also studies of Wikipedia's biased
attitudes to what edits may stay in an article. The part of our study
that concerned talk-pages and arbitration requests contributed to the
latter line of enquiry.

And we believe that this may not merely be a problem native to
Wikipedia, or on-line discourse. We believe that Wikipedia may merely
present a more-than-usually analysable trace of the inter-personal
relations that shape all collaborations. In some of the cases
described here these interactions are evidenced in talk-page
communication. In others, like in undo-redo edit wars, the interaction
is embedded in the act of editing itself. 

And this is not to mention the myriad external factors that also
affect regard to individual edits -- long-standing off-Wikipedia
arguments in both the Rupert Sheldrake and Derek Smart articles
referred to previously are good examples. Again, with search engines
we have the opportunity to track these waves of opinion -- complex
changes in the standard of knowledge of that occur naturally.

But the individual predilections of key, more-active editors, if not
evidenced in previous edits, remain a mystery. As may, of course, the
habits and opinions of a manger in a group-coding scenario, until
they are codified into action.

The major lesson we learn overall, then, is that if we can't predict
the fore-coming trajectory of a work, we may certainly have the means
to analyse it once the fact of the history of the work is crystallised
into its final form. We achieved this by deciding the article was
conceptually `finished' at the most recent revision we fetch. The
`finished' state, however, occurs naturally in other forms of work.

If surrounding context as important as it seems in this study, then we
have learnt that a user's stake in a collaborative work should be as
affected by the degree of discussion and coordination that that
individual has engaged in as much as it is by the remaining artefacts
of his or her contributions extant in the work's final form.

In the chaos of on-line interaction that characterises Wikipedia, we
must look not only to the final work. We must take into account
measures of interaction, and a users weight in the overall community
network of the on-line space. Their valuable contributions are not
only textual, but social; the collaboration is not merely the sum of
its parts, but also product of a community, a network, and a society.

%% one
%% of the most popular websites in the world, we find that other factors
%% are more correlated with survival -- factors incident to to the nature
%% of the Wikipedia `community', and a user's place within that
%% community, rather than the content of the edit itself.

%% We have looked into the possibility towards the idea of quality in
%% Wikipedia. 

%% We saw that many MANY studies use many seemingly arbitrary means to
%% find GOOD in a wikipedia.

%% We wanted something more generic. There were studies that used less
%% arbitrary means. CITE crowdsourced WikiVandalism corpus. 

%% So we considered looking at the current point of an article to be a
%% kind of culmination point. This seemed appropriate for Wikipedia's
%% philosophy. See quote. But also for the nature of knowledge itself --
%% we learn new things, we incorporate them. A conclusion can keep being
%% added to.

%% ALSO MAYBE just maYBE an article has value in the past. but we're not
%% there.
%% -
%% In the end we are seemingly brought to the conclusion that naive
%% textual analysis does not affect survival rate of text -- there are
%% simply too many contextual factors governing it. From our failure to
%% find correlation between our gradient factor and our other dat (see
%% section~\ref{mlisbad}), we may begin to conclude that text type isn't
%% innately connected to 

%% So, with our automatically we are unable to predict the shape of the
%% trajectory graph until the entire history is fetched. Does that mean
%% this is a failure? 

%% There is a hope we can predict success on edit count, `seniority' of
%% editor. We could perhaps analyse the edit with awareness, looking for
%% corroboration online with the information contained in the
%% text. Way beyond the scope of this project, but possible.

%% However, we can't predict the future. And if we can predict the
%% edit-power of an editor, we can't necessarily predict his opinion. Or
%% even his mood at the time of editing. Refer to Rupert Sheldrake.


%% INSERT THING ABOUT GOOGLE POPULAR THING

%% In the world of physical collaboration these kind of interpersonal
%% artefacts of cooperation are apparent. With wikipedia we get a clearly
%% traceable stuff.

%% PLOT DISTRIBUTION OF EDIT NUMBER ON EN, DE, IT

%% PLOT DISTRIBUTION OF SPECIES OF TEXT 

%% THE CASE FOR STUDYING NETWORK 

%% --Frederick Chopin + talk
%% --Rupert Sheldrak + talk
%% --Derek Smart + talk
%% --Aluminium + talk

%% \url{https://en.wikipedia.org/wiki/Wikipedia:Harassment#User_space_harassment}
%% \url{https://en.wikipedia.org/wiki/Wikipedia:Wikipedia_is_in_the_real_world}
%% \url{https://en.wikipedia.org/wiki/Wikipedia:Wikipedia_is_anonymous}
%% \url{https://en.wikipedia.org/wiki/Wikipedia:Harassment#Wikihounding}
