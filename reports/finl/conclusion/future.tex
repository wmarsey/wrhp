\subsection*{Summary}
With these latter cases we see the effects of things external to the
articles affecting their content, and how we may bring these things to
bear on the calculation of share. 

It is perhaps one of the most interesting parts of Wiki analysis that
evidence such interpersonal acts is so readily available. Due to the
nature of the WikiMedia software, these interactions are as publicly
available as the article's themselves, all utilising the same text
revision framework. 

Though, we are concious that in other forms of collaboration -- for
which the Wiki dataset is standing in for proof-of-concept study --
these interactions are less readily available. In cases of multi-user
source control, these interactions may occur over personal emails, on
private forums, or in the physical space of a shared workspace.

In this way then, the Wikipedia dataset is almost too complete. By
tunnelling all activity through a common system, we are able to
discover all organisation that surround the creation of the
collaborative works. And while we see that these interactions give
much greater context to collaborative contributions, we must
understand that this kind of data is unavailable on some platforms.

Our findings corroborate with existing research on individual user
activity in large-scale collaboration platforms. 

It has been found that the population and dynamics of an online space
may affect individual user behaviour. Jones, Ravid, Rafaeli find that
Newsgroups users publish shorter posts in more populous threads.

INSERT THING ABOUT GOOGLE POPULAR THING

%% In the world of physical collaboration these kind of interpersonal
%% artefacts of cooperation are apparent. With wikipedia we get a clearly
%% traceable stuff.

%% PLOT DISTRIBUTION OF EDIT NUMBER ON EN, DE, IT

%% PLOT DISTRIBUTION OF SPECIES OF TEXT 

%% THE CASE FOR STUDYING NETWORK 

%% --Frederick Chopin + talk
%% --Rupert Sheldrak + talk
%% --Derek Smart + talk
%% --Aluminium + talk

%% \url{https://en.wikipedia.org/wiki/Wikipedia:Harassment#User_space_harassment}
%% \url{https://en.wikipedia.org/wiki/Wikipedia:Wikipedia_is_in_the_real_world}
%% \url{https://en.wikipedia.org/wiki/Wikipedia:Wikipedia_is_anonymous}
%% \url{https://en.wikipedia.org/wiki/Wikipedia:Harassment#Wikihounding}
