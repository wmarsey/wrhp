EVALUATE METHODS YO \label{split-distance-eval}

We may consider other ways of looking at the data we have
computed. The gradient factor that we calculate describes how much a
given revision changes in terms of the final state -- it is a real
number from 0 to 1, 0 being a `perfect' move away from the final
version, 1 being a `perfect' move towards. On the trajectory graphs
discussed these two values are represented by a vertical line between
two points, going upwards and downwards, respectively. 

We consider that we could perhaps use it to characterise a 'perfect
edit' -- one that approaches the final edit most efficiently; in a
linear fashion. An optimum revision history may be considered as a
straight path from origin to destination. No edit inserted text that
was later removed, and the approach to the final version was as
efficient as possible over time. We may characterise such an optimum
gradient as follows:


\[
  gfactor_{optimum} = gfactor(ed(rev_{0_{content}},
  rev_{n_{content}}), rev_{n_{tstamp}} - v_{0_{tstamp}})
\]


\textbf{Reversals.} The Damerau-Levenshtein distance defines an `swap'
operation, which is the reversal of two adjacent characters. It is
particularly suited to spell-checking, and for analysing DNA-sequence
variations. In this case:\\ $ed_{damerau}(\text{``ab''},\text{``ba''})
= 1$

\textbf{Block distance.} This allows for displacements of entire
blocks to count as one operation. For
example:\\ $ed_{block}(\text{``abcde''},\text{``cdeax''})= 2$ \\ One
move of the block `cde', one substitution of `b' for
`x'.\cite{Tichy1984}

\subsection*{Density of edit}

If we have a set of the indexes of an edit operation as
$\{op_0,op_1,op_2,\dots, op_n\}$, where $op_i$ is the index of the
$i$th operation, then we may evaluate it's density with a standard
deviation of the edit itself, $\sigma_{ed}$, multiplied by the span of
the edit itself in context of the wider article, and some weighting
factor k. Something along the lines of:
$$ed_{density} = k\bullet\frac{(op_n - op_0)\sigma_{ed}}{|v_{ed}|}$$
where $|v_{ed}|$ is the overall length of resultant version. By
implementing this carefully, we may achieve a gradient of weighting,
with a lower weight values for things like spell-checks, and higher
values for whole-paragraph changes.

\textbf{\textit{q}-grams distance.} \textit{q}-grams are simply
sub-strings, and this measure describes the similarity of two strings
in terms of \textit{q}-grams they
share.\cite{Ukkonen1992}\\ $ed_{q-gram}(x,y)=\sum\limits_{v\in\Sigma
  ^q}|G(x)[v]-G(y)[v]|$\\ where $G(x)[v]$ returns the number of
occurrences of \textit{q}-gram v in string x, and $\Sigma ^q$ is all
the possible \textit{q}-grams in the alphabet (capped by string
length). $|G(x)[v]-G(y)[v]|$ a large positive number every time a
\textit{q}-gram appears a large amount of times in one string, but not
the other; it returns 0 if the substring apears the same number of
times. So, the whole function measures this difference for all
possible substrings, and sums them, returning a high number for
difference, and a low number for similarity.


Other algorithms we may look at are those that, like the
\textit{q}-gram distance, principally concern themselves with finding
common subsequences between the strings. The common subsequence
problem relates to the editdistance problem by way of the
heuristic that two similar strings will have similar subsequences ---
the \textit{q}-gram algorithm, for instance, relies on this heuristic,
and works well for most texts, it does not agree with all distance
measures. For example, two strings that are very different according
to this heuristic may be quite similar according to the
Damrau-Levenshtein measure.


\subsection*{Delta encoding}
Finally, we may also look into Delta encoding algorithms. These
describe ways of compressing the storage of a document's history --- a
format in which only the differences between each text is stored, not
the entire version. These algorithms are of the same family of
algorithms discussed above. In fact, we find that one of the fastest
known algorithms,\footnote{According to Hunt's 1998
  study\cite{Hunt1998}} \textit{VDelta}, is a refinement of the block
distance algorithm mentioned above. For a given version $n$ of a
document $doc$ is defined as:

$$v_n = v_0 \cup {\Delta}(v_0,v_1) \cup {\Delta}(v_1,v_2) \cup \dots
\cup {\Delta}(v_{n-1},v_n) $$

where ${\Delta}(v_i,v_j)$ is the difference between version $i$ and
version $j$ of the document, and the union operation $\cup$ combines
each version in a manner particular to the $\Delta$ data-type. Storing
data in this way can be very efficient, resulting in a compression
factors of five or ten on typical data.\cite{Macdonald2000} It may
also be relatively easy to maintain in our case, due to the linear
nature of Wikipedia revision histories.
