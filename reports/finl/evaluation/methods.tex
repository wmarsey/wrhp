\section{Limitations, optimisations and extensions}
\subsection*{Alignment of levenshtein distance}
\label{split-distance-eval}
As we began to realise that, in terms of survival rate, the details of
the content were somewhat secondary to the context in which those
edits were made, we began to wonder if we could make the data analysis
more efficient by doing away with the pair distance calculation.

Further than that, however, we wondered whether we could find the
distance between each revision by just taking the absolute distance
between each trajectory distance -- i.e. the distance-from-final at
each point. We saw no theoretic problem with the measure, and set out
to see if the theory that

\[
  |(ed(rev_i,rev_x) - ed(rev_j,rev_x))| = ed(rev_j,rev_i)
\]

(i.e. that an edit distance relation is transitive) could be observed
in the edits. 

Running the SQL statements found in appendix~\ref{sec:sql-things}, we
found evidence contrary to our theory. We found that only around 30\%
of our different measures matched up, with a high average mis-match of
22,181. The mismatch is, on average, around 40\% of the larger
distance.

We propose two theories for this. We can identify a problem with our
string splitting -- it may mis-align two strings, creating non-minimum
calculations, such as in the diagram on page~\pageref{fig:sub-opt}. We
illustrate the problem in figure~\ref{fig:split-string-problem}.

We can also imagine that it may be more efficient to swap some
characters between a and c, rather than delete to them to get from a
to b, and reinsert from b to c. This may also be described as an
alignment problem.

Whilst inaccurate, doing away with pair-distance calculations may be
acceptable as a rough measure, and if pair-distance, as we suspect, is
not a very effective measure of survival.

\begin{figure}[p]
  \centering
  \begin{subfigure}[t]{\linewidth}
    \begin{tikzpicture}[
        block1/.style={
          text width=3cm, 
          minimum height=2cm,
          align=center,
          anchor=north,
        },
        block2/.style={
          text width=3cm, 
          minimum height=1.5cm,
          align=center,
          anchor=north,
        },
        font=\small
      ]
      \node (a) [block1] {We can use a [[spork]] to eat spaghetti.};

      \node (b) [block1, right=1cm of a]{We can use forks to eat lovely
        spaghetti.};

      \node (c) [block1,below=1cm of a]{We can use a \hilight{[[spork]]} to eat
        spaghetti.};

      \node (d) [block1,below=1cm of b]{We can use forks to eat lovely
        spaghetti.};

      \node (e) [block2,below left=1.5cm and 1.5cm of c] {\hilight{[[spork]]}};
      
      \node (f) [block2,right=1cm of e] {We can use a\ \  to eat spaghetti.};

      \node (g) [block2,right=1cm of f] {$\emptyset$};
      
      \node (h) [block2,right=1cm of g] {We can use forks to eat lovely
        spaghetti.};

      \node (i) [block2,below=1cm of e] {\hilight{[[spork]]}}; 
      
      \node (j) [block2,below=1cm of f] {$\emptyset$}; 
      
      \node (k) [block2,below =1cm of g] {We can use a\ \  to eat
        spaghetti.};

      \node (l) [block2,below =1cm of h] {We can use forks to eat lovely
        spaghetti.};

      \node (m) [block2,below= 1cm of i] {$ed_{links} = 9$};

      \node (n) [block2,below =1cm of k] {$ed_{normal} = 18$};
      
      \node (o) [block2,below left=1.5cm and 1.5cm of n] {$ed_{total} = 27$};

      \draw [->] (a) -- (c);
      \draw [->] (b) -- (d);
      \draw [->] (c) -- (e);
      \draw [->] (c) -- (f);
      \draw [->] (d) -- (g);
      \draw [->] (d) -- (h);
      \draw [->] (e) -- (i);
      \draw [->] (g) -- (j);
      \draw [->] (f) -- (k);
      \draw [->] (h) -- (l);
      \draw [->] (i) -- (m);
      \draw [->] (j) -- (m);
      \draw [->] (k) -- (n);
      \draw [->] (l) -- (n);
      \draw [->] (m) -- (o);
      \draw [->] (n) -- (o);

    \end{tikzpicture}
    \caption{Split string edit distance}
  \end{subfigure}\\
  \vspace{20mm}
  \begin{subfigure}[t]{\linewidth}
    $ed($``We can use a [[spork]] to eat spaghetti.''$,$``We can use forks
    to eat lovely spaghetti.''$) = 6$
    \caption{Whole-string edit distance}
  \end{subfigure}\\
  \vspace{20mm}
  \caption{Showing alignment alignment inefficiency in split-string
    Levenshtein distance processing.}
\label{fig:split-string-problem}
\end{figure}

\subsection*{Calculating `turbulence'}
The gradient factor that we calculate describes how much a given
revision changes in terms of the final state -- it is a real number
from 0 to 1, 0 being a `perfect' move away from the final version, 1
being a `perfect' move towards. On the trajectory graphs discussed
these two values are represented by a vertical line between two
points, going upwards and downwards, respectively.

We consider that we could perhaps use it to characterise a 'perfect
edit factor' -- one that approaches the final edit most efficiently; in a
linear fashion. An optimum revision history may be considered as a
straight path from origin to destination. No edit inserted text that
was later removed, and the approach to the final version was as
efficient as possible over time. We may characterise such an optimum
gradient as follows:

\[
  gfactor_{optimum} = gfactor(ed(rev_{0_{content}},
  rev_{n_{content}}), rev_{n_{tstamp}} - rev_{0_{tstamp}})
\]

This equation gives a gradient that, if the same for every edit, would
describe the most stable accumulation of data, with no peaks or
fluctuations in activity. The average deviation from this number can
measure the turbulence of an article's path. 

We may also look to density of edit. If we have a set of the indexes
of an edit operation as $\{op_0,op_1,op_2,\dots, op_n\}$, where $op_i$
is the index of the $i$th operation, then we may evaluate it's density
with a standard deviation of the edit itself, $\sigma_{ed}$,
multiplied by the span of the edit itself in context of the wider
article. Something along the lines of:
$$ed_{density} = \frac{(op_n - op_0)\sigma_{ed}}{|v_{ed}|}$$ where
$|v_{ed}|$ is the overall length of resultant version. By implementing
this carefully, we may achieve a gradient of weighting, with a lower
weight values for things like spell-checks, and higher values for
whole-paragraph changes.

However, this density is somewhat reflected in the gradient factor
measure, as discussed before.

\subsection*{Levenshtein algorithm changes}
The Damerau-Levenshtein distance defines an `swap' operation, which is
the reversal of two adjacent characters. It is particularly suited to
spell-checking, and for analysing DNA-sequence variations. In this
case:

$ed_{damerau}(\text{``ab''},\text{``ba''}) = 1$

Implementing this could more accurately define spell-check operations.

Block distance operation would allow us to recognise displacements of
entire blocks to count as one operation. For
example:

$ed_{block}(\text{``abcde''},\text{``cdeax''})= 2$

One move of the block `cde', one substitution of `b' for
`x'.\cite{Tichy1984} We could implement this in order to better
diagnose large movements of text --- perhaps between documents. (The
Harry Potter history in appendix~\ref{sec:traj-article-talk} shows
frequent migrations of text to book-, character- and film-specific
pages.)

\subsection*{Awarding restructuring}
\label{restructuring}
It has been found that, even in the most accurate articles, that the
structure of Wikipedia article can be weak.\cite{Giles2005} We made
some efforts to recognise attempts at structural change previously with our regex splitting system. However, these attempts are limited. We really
need to be able to identify the move of large blocks of text -- not
only inside an article, but to others. With a larger dataset and the latter idea
necessary computing power this would just be a matter of
cross-referencing the deletion of largish chunks of text with
insertions of similar text into other articleswithin a short time frame.

\subsection*{Refinement of gradient factor}
A limitation of gradient factor is it's reliance upon a relatively
arbitrary measure of time. Though the measure of time has been
consistent throughout the study, the measurement does affect the
distribution of possible gradient factor values to the extremities of
the range. It would be recommended in further research to take this
into account. If someone makes an edit, commits it, fixes spelling,
and commits again, does he receive less reward than if he had
submitted at once? Or more? This is something we wished we would have
had time to look into during the study.

\subsubsection*{Extending data}
Grabbing each page as we went was useful for cleaning and sometimes
preprocessing parts of the data, but the fetches were time
consuming. Also, when we came to look for more general data about
individual users, and tried to find inter-article connections, we had
to do so in the knowledge that our databases represented a very small
selection from the Wikipedia's in question.

For further work on could be better off using a wikipedia dump. These
are made monthly and represent the entirety of the site, history
included at the point of export. They are, however, $~800$GB,
compressed.\cite{wiki-dump} A past studies used a hadoop framework
to enable the study of this data. In the study here we saved time by
simply setting up the API access, but perhaps further study could
invest time in setting up a more complete dataset.

\subsubsection*{Further subjects}
As mentioned previously, this project may easily expand beyond the
wikipedia dataset. A git project history, for instance, may be of
interest for further study. We may fairly easily combine the existing
research with metrics that concern code in particular, such as
Cyclomatic Complexity, which measures code flow complexity according
to its logical operators.\cite{McCabe1976} The main challenge on
platforms would be mining the contextual data that was so easily found
on Wikipedia.
