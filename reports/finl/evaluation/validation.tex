\section{Machine learning validation attempts}
\label{mlisbad}
We attempted to validate our model using machine learning. Given the
form of the model's output -- a 1D array of numbers -- machine
learning seemed to be a natural choice.

We were most interested if we could predict gradient factor. Since
this is a real number measurement of whether or not the edit was
included in the final version, we were interested if a machine
learning could possibly predict the gradient factor given the weights
that we calculate. We could possibly see if the data we had collected
would be able to similarly measure affects on the article.

We used the scikit-learn python package to approach the
problem,~\ref{scikit-learn} working with the ~80,000 test cases extant
in the database at the time. We tried predicting gradient factor with
the following forms of training data:

\begin{itemize}
  \item \textbf{Weights}\\
    Simply the calculated weights, as found in the database weight table. 

  \item \textbf{Weights and size change}\\
    The calculated, along with information about whether the revision
    made the article larger or smaller

  \item \textbf{Weights size change, time change}\\
    The calculated, along with information about whether the revision
    made the article larger or smaller

  \item \textbf{Weights summed to one value, and size change}\\
    Summing the weights to one value is roughly equivalent to taking a
    plain levenshtein distance of the article on the whole, rather
    than one separated by species.
    
  \item \textbf{Summed weights and user edit count over
    domain}\\ We take the generic weight and combine it with the
    user's activity over the whole wikipedia domain.
    

  \item \textbf{Summed weights and user edit count over
    article}\\ Similar to above, but only counts user activity in the
    same article.

\end{itemize}

We experimented with various types of learning, including the blah and
blah. Appendix --> typical  output script

Results not good. The regression models failed to fit to the data.

Since the gradient factor is sensitive to time, and the scale of that
time, we also tried seperating the data into two categories,
'towards-final' as 1 and 'away-from-final' as 0, by simply rounding
the gradient factor to it's nearest integer, and repeated the same
tests as detailed above.

Not good.

Two problems. The way we collect our data, or text doesn't matter

Why data collection could be a problem. talk about lack of talk-page
data, refer to analysis. Please see appendix of plots of page to
talk-page. But slowness of stuff, sad. 

But then talk about all the research that says uknown edits are
undone. 

Talk about how editors are down since 2007.

Talk about how this is trying to be undone by WikiMedia, but for the
moment, more contextual.

ALSO, complicating the wikipedia case. like the things in analyses,
talk about completely external things. Like celebrity deaths.
