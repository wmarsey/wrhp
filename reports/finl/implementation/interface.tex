\section{The interface}
The CLI can be accessed using the shortcut wrhp (Wiki Revision History
Portal...). The arguments are as follows:

\begin{itemize}[label={}]
  \item \textbf{Default behvaiour.} Default behaviour is -s.
  \item \textbf{-s} Scrape and analyse. Attempts to fetch a new page
    to the database, and process it. If used with -p or -i, means to
    attempt to collect a new page from the site before plotting,
    rather than taking one from the database.
  \item \textbf{-p} Plot data. Saves png files to location given by
    the --plotpath argument.
  \item \textbf{-i} Open the interactive plot window for a given (or
    random) article once analysed.
  \item \textbf{-v} View a Wikipedia page online. Must be used with
    --domain. Can be used to view a diff (using --revid and
    --oldrevid), a specific revision (using only --revid), or the
    latest version of a page (using --pageid).
  \item \textbf{-t} `Trundle' mode. Repeats the given operation until
    interrupted. Useful for building up a database. Cannot be used
    with the --titles argument.
  \item \textbf{--title [\textit{str}]} Specify the
    pages to be scraped. Must be used with --domain. Case (and
    spelling...)  sensitive.
  \item \textbf{--domain [\textit{str}]} Specify the domain to connect
    to. May be used without --titles, limiting the random page pick to
    one domain. Must be the short version (`en', `de', etc.).
  \item \textbf{--scrapemin [\textit{int}]} Specify the minimum amount of
    pages to be scraped for one page. Default = 50.
  \item \textbf{--plotpath [\textit{filepath}]} Specify location for
    plots to be stored. Default = .
  \item \textbf{--revid} Specify the revision ID to be viewed. Used
    with -v.
  \item \textbf{--oldrevid} Specify the old revision ID when viewing a
    diff. Used with -v/
  \item \textbf{--pageid} Specify a target pageid. Can be used in -v
    and -s.
\end{itemize}
