\section{Technology}
Out first stop was to implement a library for tracing a Wikipedia
history, harnessing the Wikipedia API in order to download information
about articles and their histories, as well as their content. It is
inspired by open Wikipedia metadata classes such as `Wikipedia
Miner'\cite{wiki-miner}, or the revision-fetching `Java Wikipedia
Library / Wikipedia Revision
Library'.\cite{wiki-java}\cite{Ferschke2011} The python package
`wikipedia',\cite{python-wikipedia} was also a useful starting point,
but was not appropriate for the project in general.

The Levenshtein distance calcalator, being an intensive computation,
is implemented in C++ for speed purposes. In order to import to
Python, we prepare python variables within the C++ code, and compile
and build a shared object library as defined in the Python
documentation.\cite{python-extend-c++} To further improve speed, we
deploy each levenshtein distance calculation in its own process, to
allow for parallel processing on multi-core machines.

The database is implemented in PostgreSQL, with a python wrapper
written using the psycopg2 module.\cite{psycopg2} The text is
automatically compressed on insertion -- this is the default in
PostreSQL.\cite{psql-comp} 

The graphs are outputted using the matlab-inspired python package,
matplotlib.\cite{matplotlib} 
