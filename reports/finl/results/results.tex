The basic output of the model described here is a set of numbers, a
levenshtein distance for each of the groups described in
line~\ref{dist-calc-groups} of algorithm~\ref{dist-calc} (and
described more clearly by the schema in figure~\ref{database-schema}),
and a trajectory factor, calculated by the procedure found in
algorithm~\ref{traj-calc}. Accompanying this is the trajectory
distance of each revision.

Through simple database queries we may also find out other related
information, such as how many edits an article has, how frequent these
edits were, and the article's final or average size.

With a large enough random sample, we begin to be able to report the
relative activity of individual editors also. If, for every page `$X$'
we also look for and grab the page `Talk:$X$' we may see which editors
were involved in discussions around the edits, and begin to more
clearly see the context surrounding each article. We can also look for
pages named `Wikipedia:Requests\_for\_arbitration/$X$' to discover
evidence of more serious disputes. The CLI does not implement this
automatically, but we will discuss how to combine this information
automatically and what we find out from it.

Using the DataPlot class is built to output four different kinds of
graphs -- trajectory graphs (trajectory against article growth), a
trajectory vs talk page graph, a bar chart grading users by edit
count, and a similar graph grading them by the `share' of the
article. This latter graph can be modified by the application of
weights. The weights allow us to take some species of text as more
important than others when calculating a user's share in the article.

The DataPlot class can also output graphs on the database as a
whole. Running it as a script we can plot information about how the
size of articles in each language, the lengths of their histories,
etc.
