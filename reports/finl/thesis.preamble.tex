\usepackage{graphicx}
\DeclareGraphicsExtensions{.pdf,.jpeg,.jpg}

\usepackage{verbatim}
\usepackage{latexsym}
\usepackage{mathchars}
\usepackage{setspace}

\usepackage{cleveref} 
\usepackage{color}               %obv
\usepackage{appendix}            %obv
\usepackage{amsmath}             %for math environment

\usepackage{enumitem}            %for modifying lists
\setitemize{noitemsep,topsep=0pt,parsep=0pt,partopsep=0pt}
\setenumerate{noitemsep,topsep=0pt,parsep=0pt,partopsep=0pt}

\usepackage[procnames]{listings} %for inserting code
\usepackage{fancyvrb} % for inserting .txt
\usepackage{parskip}             %for modifying spacing

\usepackage{float}              %for forcefully placing diagrams 
\usepackage[colorlinks=false, pdfborder={0 0 0}]{hyperref}
\usepackage{algorithm}
\usepackage{algpseudocode}
\newcommand{\tab}[1]{\hspace{.08\textwidth}{#1}} % indent tab for data
                                                 % structs in algos
\newcommand{\LineIf}[3]{ {#1}
  \algorithmicif\ {#2}
  \algorithmicelse\ {#3} } % inline X if Y else Z

\usepackage{booktabs}
%\usepackage{subfig}
\usepackage{subcaption}

\usepackage[backref=true,
  %style=authoryear,
  style=numeric-comp,
  citereset=section,
  maxcitenames=3,
  maxbibnames=100]{biblatex}
\bibliography{thesis}
\DefineBibliographyStrings{english}{%
  backrefpage  = {see p.}, % for single page number
  backrefpages = {see pp.} % for multiple page numbers
}
\setlength\bibitemsep{1em}

\usepackage{titlesec}
\titlespacing\section{0pt}{12pt plus 4pt minus 2pt}{0pt plus 2pt minus 2pt}
\titlespacing\subsection{0pt}{12pt plus 4pt minus 2pt}{0pt plus 2pt minus 2pt}
\titlespacing\subsubsection{0pt}{12pt plus 4pt minus 2pt}{0pt plus 2pt minus 2pt}

\setlength{\parskip}{\medskipamount}  % a little space before a \par
\setlength{\parindent}{0pt}	      % don't indent first lines of paragraphs
%UHEAD.STY  If this is included after \documentstyle{report}, it adds
% an underlined heading style to the LaTeX report style.
% \pagestyle{uheadings} will put underlined headings at the top
% of each page. The right page headings are the Chapter titles and
% the left page titles are supplied by \def\lefthead{text}.

% Ted Shapin, Dec. 17, 1986

\makeatletter
\def\chapapp2{Chapter}

\def\appendix{\par
 \setcounter{chapter}{0}
 \setcounter{section}{0}
 \def\chapapp2{Appendix}
 \def\@chapapp{Appendix}
 \def\thechapter{\Alph{chapter}}}

\def\ps@uheadings{\let\@mkboth\markboth
% modifications
\def\@oddhead{\protect\underline{\protect\makebox[\textwidth][l]
		{\sl\rightmark\hfill\rm\thepage}}}
\def\@oddfoot{}
\def\@evenfoot{}
\def\@evenhead{\protect\underline{\protect\makebox[\textwidth][l]
		{\rm\thepage\hfill\sl\leftmark}}}
% end of modifications
\def\chaptermark##1{\markboth {\ifnum \c@secnumdepth >\m@ne
 \chapapp2\ \thechapter. \ \fi ##1}{}}%
\def\sectionmark##1{\markright {\ifnum \c@secnumdepth >\z@
   \thesection. \ \fi ##1}}}
\makeatother
%%From: marcel@cs.caltech.edu (Marcel van der Goot)
%%Newsgroups: comp.text.tex
%%Subject: illegal modification of boxit.sty
%%Date: 28 Feb 92 01:10:02 GMT
%%Organization: California Institute of Technology (CS dept)
%%Nntp-Posting-Host: andromeda.cs.caltech.edu
%%
%%
%%Quite some time ago I posted a file boxit.sty; maybe it made it
%%to some archives, although I don't recall submitting it. It defines
%%	\begin{boxit}
%%	...
%%	\end{boxit}
%%to draw a box around `...', where the `...' can contain other
%%environments (e.g., a verbatim environment). Unfortunately, it had
%%a problem: it did not work if you used it in paragraph mode, i.e., it
%%only worked if there was an empty line in front of \begin{boxit}.
%%Luckily, that is easily corrected.
%%
%%HOWEVER, apparently someone noticed the problem, tried to correct it,
%%and then distributed this modified version. That would be fine with me,
%%except that:
%%1. There was no note in the file about this modification, it only has my
%%   name in it.
%%2. The modification is wrong: now it only works if there is *no* empty
%%   line in front of \begin{boxit}. In my opinion this bug is worse than
%%   the original one.
%%
%%In particular, the author of this modification tried to force an empty
%%line by inserting a `\\' in the definition of \Beginboxit. If you have
%%a version of boxit.sty with a `\\', please delete it. If you have my
%%old version of boxit.sty, please also delete it. Below is an improved
%%version.
%%
%%Thanks to Joe Armstrong for drawing my attention to the bug and to the
%%illegal version.
%%
%%                                          Marcel van der Goot
%% .---------------------------------------------------------------
%% | Blauw de viooltjes,                    marcel@cs.caltech.edu
%% |    Rood zijn de rozen;
%% | Een rijm kan gezet
%% |    Met plaksel en dozen.
%% |


% boxit.sty
% version: 27 Feb 1992
%
% Defines a boxit environment, which draws lines around its contents.
% Usage:
%   \begin{boxit}
%	... (text you want to be boxed, can contain other environments)
%   \end{boxit}
%
% The width of the box is the width of the contents.
% The boxit* environment behaves the same, except that the box will be
% at least as wide as a normal paragraph.
%
% The reason for writing it this way (rather than with the \boxit#1 macro
% from the TeXbook), is that now you can box verbatim text, as in
%   \begin{boxit}
%   \begin{verbatim}
%   this better come out in boxed verbatim mode ...
%   \end{verbatim}
%   \end{boxit}
%
%						Marcel van der Goot
%						marcel@cs.caltech.edu
%

\def\Beginboxit
   {\par
    \vbox\bgroup
	   \hrule
	   \hbox\bgroup
		  \vrule \kern1.2pt %
		  \vbox\bgroup\kern1.2pt
   }

\def\Endboxit{%
			      \kern1.2pt
		       \egroup
		  \kern1.2pt\vrule
		\egroup
	   \hrule
	 \egroup
   }	

\newenvironment{boxit}{\Beginboxit}{\Endboxit}
\newenvironment{boxit*}{\Beginboxit\hbox to\hsize{}}{\Endboxit}
\input{icthesis.sty}
%\usepackage{algorithm2e}

\newcommand{\ipc}{{\sf ipc}}

\newcommand{\Prob}{\bbbp}
\newcommand{\Real}{\bbbr}
\newcommand{\real}{\Real}
\newcommand{\Int}{\bbbz}
\newcommand{\Nat}{\bbbn}

\newcommand{\NN}{{\sf I\kern-0.14emN}}   % Natural numbers
\newcommand{\ZZ}{{\sf Z\kern-0.45emZ}}   % Integers
\newcommand{\QQQ}{{\sf C\kern-0.48emQ}}   % Rational numbers
\newcommand{\RR}{{\sf I\kern-0.14emR}}   % Real numbers
\newcommand{\KK}{{\cal K}}
\newcommand{\OO}{{\cal O}}
\newcommand{\AAA}{{\bf A}}
\newcommand{\HH}{{\bf H}}
\newcommand{\II}{{\bf I}}
\newcommand{\LL}{{\bf L}}
\newcommand{\PP}{{\bf P}}
\newcommand{\PPprime}{{\bf P'}}
\newcommand{\QQ}{{\bf Q}}
\newcommand{\UU}{{\bf U}}
\newcommand{\UUprime}{{\bf U'}}
\newcommand{\zzero}{{\bf 0}}
\newcommand{\ppi}{\mbox{\boldmath $\pi$}}
\newcommand{\aalph}{\mbox{\boldmath $\alpha$}}
\newcommand{\bb}{{\bf b}}
\newcommand{\ee}{{\bf e}}
\newcommand{\mmu}{\mbox{\boldmath $\mu$}}
\newcommand{\vv}{{\bf v}}
\newcommand{\xx}{{\bf x}}
\newcommand{\yy}{{\bf y}}
\newcommand{\zz}{{\bf z}}
\newcommand{\oomeg}{\mbox{\boldmath $\omega$}}
\newcommand{\res}{{\bf res}}
\newcommand{\cchi}{{\mbox{\raisebox{.4ex}{$\chi$}}}}
%\newcommand{\cchi}{{\cal X}}
%\newcommand{\cchi}{\mbox{\Large $\chi$}}

% Logical operators and symbols
\newcommand{\imply}{\Rightarrow}
\newcommand{\bimply}{\Leftrightarrow}
\newcommand{\union}{\cup}
\newcommand{\intersect}{\cap}
\newcommand{\boolor}{\vee}
\newcommand{\booland}{\wedge}
\newcommand{\boolimply}{\imply}
\newcommand{\boolbimply}{\bimply}
\newcommand{\boolnot}{\neg}
\newcommand{\boolsat}{\!\models}
\newcommand{\boolnsat}{\!\not\models}

\newcommand{\op}[1]{\mathrm{#1}}
\newcommand{\s}[1]{\ensuremath{\mathcal #1}}

% Properly styled differentiation and integration operators
\newcommand{\diff}[1]{\mathrm{\frac{d}{d\mathit{#1}}}}
\newcommand{\diffII}[1]{\mathrm{\frac{d^2}{d\mathit{#1}^2}}}
\newcommand{\intg}[4]{\int_{#3}^{#4} #1 \, \mathrm{d}#2}
\newcommand{\intgd}[4]{\int\!\!\!\!\int_{#4} #1 \, \mathrm{d}#2 \, \mathrm{d}#3}

% Large () brackets on different lines of an eqnarray environment
\newcommand{\Leftbrace}[1]{\left(\raisebox{0mm}[#1][#1]{}\right.}
\newcommand{\Rightbrace}[1]{\left.\raisebox{0mm}[#1][#1]{}\right)}

% Funky symobols for footnotes
\newcommand{\symbolfootnote}{\renewcommand{\thefootnote}{\fnsymbol{footnote}}}
% now add \symbolfootnote to the beginning of the document...

\newcommand{\normallinespacing}{\renewcommand{\baselinestretch}{1.5} \normalsize}
\newcommand{\mediumlinespacing}{\renewcommand{\baselinestretch}{1.2} \normalsize}
\newcommand{\narrowlinespacing}{\renewcommand{\baselinestretch}{1.0} \normalsize}
\newcommand{\bump}{\noalign{\vspace*{\doublerulesep}}}
\newcommand{\cell}{\multicolumn{1}{}{}}
\newcommand{\spann}{\mbox{span}}
\newcommand{\diagg}{\mbox{diag}}
\newcommand{\modd}{\mbox{mod}}
\newcommand{\minn}{\mbox{min}}
\newcommand{\andd}{\mbox{and}}
\newcommand{\forr}{\mbox{for}}
\newcommand{\EE}{\mbox{E}}

\newcommand{\deff}{\stackrel{\mathrm{def}}{=}}
\newcommand{\syncc}{~\stackrel{\textstyle \rhd\kern-0.57em\lhd}{\scriptstyle L}~}

\def\coop{\mbox{\large $\rhd\!\!\!\lhd$}}
\newcommand{\sync}[1]{\raisebox{-1.0ex}{$\;\stackrel{\coop}{\scriptscriptstyle
      #1}\,$}}

\newtheorem{definition}{Definition}[chapter]
\newtheorem{theorem}{Theorem}[chapter]

\newcommand{\Figref}[1]{Figure~\ref{#1}}
\newcommand{\fig}[3]{
  \begin{figure}[!ht]
    \begin{center}
      \scalebox{#3}{\includegraphics{figs/#1.ps}}
      \vspace{-0.1in}
      \caption[ ]{\label{#1} #2}
    \end{center}
  \end{figure}
}

\newcommand{\figtwo}[8]{
  \begin{figure}
    \parbox[b]{#4 \textwidth}{
      \begin{center}
        \scalebox{#3}{\includegraphics{figs/#1.ps}}
        \vspace{-0.1in}
        \caption{\label{#1}#2}
      \end{center}
    }
    \hfill
    \parbox[b]{#8 \textwidth}{
      \begin{center}
        \scalebox{#7}{\includegraphics{figs/#5.ps}}
        \vspace{-0.1in}
        \caption{\label{#5}#6}
      \end{center}
    }
  \end{figure}
}


\usepackage{tikz}
\usetikzlibrary{shapes,
  arrows,
  chains,
  matrix,
  positioning,
  fit,
  scopes,
  calc,
  decorations.pathmorphing}
\makeatletter
\tikzset{join/.code=\tikzset{after node path={%
\ifx\tikzchainprevious\pgfutil@empty\else(\tikzchainprevious)%
edge[every join]#1(\tikzchaincurrent)\fi}}}
\makeatother
\tikzset{>=stealth',every on chain/.append style={join},
         every join/.style={->},
         snake it/.style={decorate, decoration=snake}}
\tikzstyle{labeled}=[execute at begin node=$\scriptstyle,
   execute at end node=$]

\definecolor{redi}{RGB}{255,38,0}
\definecolor{yellowi}{RGB}{255,251,0}
\definecolor{greeni}{RGB}{166,247,166}

\tikzset{ 
  table/.style={
    matrix of nodes,
    row sep=-\pgflinewidth,
    column sep=-\pgflinewidth,
    nodes={rectangle,draw=black,text width=3ex,align=center},
    text depth=0.5ex,
    text height=2ex,
    nodes in empty cells
  },
  texto/.style={font=\large\sffamily},
  title/.style={font=\large\sffamily}
}


\newcommand\CellText[2]{%
  \node[texto,left=of mat#1,anchor=east]
  at (mat#1.west)
  {\large #2};
}

\newcommand\SlText[2]{%
  \node[texto,left=of mat#1,anchor=west,rotate=50]
  at ([xshift=1.5ex,yshift=1ex]mat#1.north)
  {\large #2};
}

\newcommand{\super}[1]{\textsuperscript{#1}}

%% \renewcommand{\cite}[1]{\footcite{#1}}

\definecolor{keywords}{RGB}{255,0,90}
\definecolor{comments}{RGB}{0,0,113}
\definecolor{red}{RGB}{160,0,0}
\definecolor{green}{RGB}{0,150,0}
\lstset{frame=tb,
  language=Python,
  aboveskip=3mm,
  belowskip=3mm,
  showstringspaces=false,
  columns=flexible,
  basicstyle={\small\ttfamily},
  numbers=none,
  numberstyle=\tiny\color{gray},
  keywordstyle=\color{keywords},
  commentstyle=\color{comments},
  stringstyle=\color{red},
  breaklines=true,
  breakatwhitespace=true,
  tabsize=3,
  procnamekeys={def,class}
}
